\documentclass{article}

%\usepackage[a4paper, total={6in, 8in}]{geometry}

\usepackage{amsmath}
\usepackage{amsfonts}
\usepackage{amssymb}
\usepackage[T1, T2A]{fontenc}
\usepackage[utf8]{inputenc}
\usepackage[english, russian]{babel}
\usepackage{graphics}

\author{Муницина Валерия Александровна}
\title{Аналитическая механика}

\begin{document}
  \maketitle
  \section*{Кинематика точки}
  \paragraph*{Материальная точка} - точка, размером которой можно пренебречь.
  
  Мы будем полагать, что время меняетсяр равномерно и непрерывно.
  \begin{picture}(100, 100)
  \put(0,0){\vector(1,1){10}} %
  \end{picture}
  \subsection*{Векторное описание движения}.
  Зависимость координат от времени назовем законом движения.
  $$ \overrightarrow{r} = \overrightarrow{r}(t) \in C^2 $$
  $$ \gamma = \{ \overrightarrow{r}(t),~ t \in (0,~ +\infty) \} $$
  $$ \overrightarrow{v} = \frac{d\overrightarrow{r}}{dt} $$
  $$ \overrightarrow{w} = \frac{d\overrightarrow{v}}{dt} = \frac{d^2\overrightarrow{r}}{dt^2} $$
  \subsection*{Декартовы координаты}
  $$ \overrightarrow{r}(t) = x(t)\overrightarrow{e_x} + y(t)\overrightarrow{e_y} + z(t)\overrightarrow{e_z} $$
  $$ \overrightarrow{v}(t) = x'(t)\overrightarrow{e_x} + y'(t)\overrightarrow{e_y} + z'(t)\overrightarrow{e_z} $$
  \subsection*{Движение по окружности}
  $$ x = R cos \varphi $$
  $$ y = R sin \varphi $$
  $$ x' = -R sin \varphi \varphi' $$
  $$ y' = -R cos \varphi \varphi' $$
  $$ x'' = -R cos \varphi \varphi'^2 - R sin \varphi \varphi'' $$
  $$ y'' = -R sin \varphi \varphi'^2 + R cos \varphi \varphi'' $$
  $$ \overrightarrow{v} = 2\varphi'(-sin \varphi \overrightarrow{e_x} + cos \varphi \overrightarrow{e_y} = R \varphi' \overrightarrow{r} $$
  $$ w = R \varphi''( - sin \varphi \overrightarrow{e_x} + cos \varphi \overrightarrow{e_y}) + R \varphi'^2(-cos \varphi \overrightarrow{e_x} - sin \varphi{\overrightarrow{e_y}}) = R \varphi'' \overleftarrow{\tau} + R \varphi'^2 \overrightarrow{n} $$
  
  $$ \overrightarrow{v} = R\varphi' \overrightarrow{\tau} $$
  $$ \overrightarrow{w} = R \varphi'' \overrightarrow{\tau} + R \varphi'^2 \overrightarrow{n} $$
  
  \subsection*{Естественное описание движения}
  Кривая задана параметрически естественным параметром $s$. $ ds = |dr\overrightarrow{dr}| \neq 0 $
  Определение: 
  $$ \overrightarrow{\tau} = \frac{d\overrightarrow{r}}{ds} = \overrightarrow{r}' \text{Касательный вектор} $$
  $$ \overrightarrow{n} = \frac{\overrightarrow{r}'}{|\overrightarrow{\tau}'|} \text{ - вектор главной нормали }$$
  $$ \overrightarrow{b} = [\overrightarrow{t}; \overrightarrow{n}] \text{ - вектор бинормали }$$

  \paragraph{Утверждение} $ \{\overrightarrow{\tau}, \overrightarrow{n}, \overrightarrow{b} $ - тройка ортогональных единичных векторов.
  
  $$ |\overrightarrow{\tau}| = \frac{|d\overrightarrow{r}|}{|ds|} = 1 $$
  $$ |\overrightarrow{n}| = \frac{|\overrightarrow{r}'|}{|\overrightarrow{\tau}'|} = 1 $$
  
  Этот трехгранник называют репер Ферне. (Дарбу, сопровождающий трехгранник).
  
  \paragraph{Теорема} $ \overrightarrow{v} = v \overrightarrow{\tau} $, $ \overrightarrow{w} = v' \overrightarrow{\tau} + \frac{v^2}{\rho} \overrightarrow{n} $, где $ v = s' $.
  
  $$ \overrightarrow{v} = \frac{d\overrightarrow{r}}{dt} = \frac{d\overrightarrow{r}}{ds} \frac{ds}{dt} $$
  
  $$ \overrightarrow{\tau}' = \frac{d\overrightarrow{\tau}}{ds} \frac{ds}{dt} = \overrightarrow{n}kv $$
  $k$ - кривизна.
  $$ \overrightarrow{w} = \overleftarrow{v}' = v' \overrightarrow{\tau} + v \overrightarrow{\tau}' = v' \overrightarrow{\tau} + v^2 k \overrightarrow{n} = v' \overrightarrow{\tau} + \frac{v^2}{\rho} \overrightarrow{n} $$
  
  $$ v' \overrightarrow{\tau} \text{ - касательное ускорение} $$
  $$ \frac{v^2}{\rho} \overrightarrow{n} \text{ - нормальное ускорение } $$
  
  \paragraph{Формулы Френе:}
  $$ \overrightarrow{\tau}' = k \overrightarrow{n} $$
  $$ \overrightarrow{n}' = - k\overrightarrow{\tau} + \kappa \overleftarrow{b} $$
  $$ \overrightarrow{b}' = -\kappa\overrightarrow{n} $$
  
  $$ | \overrightarrow{n} | = 1 \Rightarrow (\overrightarrow{n}, \overrightarrow{n}) = 0 $$
  $$ \overrightarrow{n} \perp \overrightarrow{\tau} = 0 \Rightarrow (\overrightarrow{n}', \overrightarrow{\tau}) + (\overleftarrow{n}, \overrightarrow{\tau}) + k = 0 $$
  
  $$ \overrightarrow{b}' = [\overrightarrow{r}', \overrightarrow{n}] + [\overrightarrow{\tau}, \overrightarrow{n}'] = [k\overrightarrow{n}, \overrightarrow{n}] + [\overrightarrow{\tau}, -k\overrightarrow{\tau} + \kappa \overrightarrow{b}] = 0 + \kappa[\overrightarrow{r}, \overrightarrow{b}] = -\kappa\overrightarrow{n} $$
  
  \subsection*{Ортогональные векторные координаты}
  
  \begin{gather}
  \overrightarrow{r} = \overrightarrow{r}(q_1(t), q_2(t), q_3(t)) \\
  \overrightarrow{v} = \sum \limits_{i = 1}^3 \frac{\partial\overrightarrow{r}}{\partial q_i} q_i'\\
  \overrightarrow{H_i} = \frac{\partial\overrightarrow{r}}{\partial q_i} = H_i \overrightarrow{e_i} 
  \end{gather}
  $H_i$ - коэффициенты Ламе.
  \begin{gather}
  \overrightarrow{H_i} = \frac{\partial \overrightarrow{r}}{\partial q_i} = \sqrt{\left(\frac{\partial x}{\partial q_i}\right) ^2}
  \end{gather}
  Копоненты вектора ускорения в ортогональном криволинейном базисе определяются равенством:
  $$ w_i = \frac{1}{H_i}\left(\frac{d}{dt} \frac{\partial}{\partial q_i'} \left(\frac{v^2}{2}\right) - \frac{\partial}{\partial q_i} \left(\frac{v^2}{2} \right) \right) $$
  
$$ (\overrightarrow{w}, \overrightarrow{H_i}) = \left(\frac{d\overrightarrow{v}}{dt}, \frac{\partial \overrightarrow{r}}{\partial q_i} \right) = \frac{d}{dt} \left(\overrightarrow{v}, \frac{\overrightarrow{r}}{\partial q_i}\right) - \left((\overrightarrow{v}, \frac{d}{dt} \frac{\partial \overrightarrow{r}}{\partial q_i} \right) $$
$$ \frac{\partial \overrightarrow{r}}{\partial q_i} = \frac{\partial \overrightarrow{v}}{\partial q_i'} \text{ - из определения скорости} $$  
$$ \frac{d}{dt} \left(\frac{\partial \overrightarrow{r}}{\partial q_i} \right) = \sum \limits_{j = 1}^3 \frac{\partial^2 \overrightarrow{r}}{\partial q_j \partial q_i} q_j' = \sum \limits_{j = 1}^3 \frac{\partial^2 \overrightarrow{r}}{\partial q_i \partial q_j} q_i' = \frac{\partial}{\partial q_i} \left( \frac{d\overrightarrow{r}}{dt} \right) = \frac{\partial \overrightarrow{r}'}{\partial q_i} = \frac{\partial \overrightarrow{v}}{\partial q_i} $$

$$ \frac{d}{dt} \left(\overrightarrow{v}, \frac{\partial \overrightarrow{v}}{\partial q_i} \right) - \left( \overrightarrow{v}, \frac{\partial \overrightarrow{v}}{\partial q_i} \right) = \frac{d}{dt} \frac{1}{2} \frac{\partial}{\partial q_i} (\overrightarrow{v}, \overrightarrow{v}) - \frac{1}{2} \frac{\partial}{\partial q_i} (\overrightarrow{v}, \overrightarrow{v}) = \frac{d}{dt} \frac{\partial}{\partial q_i} \left(\frac{v^2}{2} \right) = \ldots $$

\end{document}

