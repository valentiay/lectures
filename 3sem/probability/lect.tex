\documentclass{article}

%\usepackage[a4paper, total={6in, 8in}]{geometry}

\usepackage{amsmath}
\usepackage{amsfonts}
\usepackage{amssymb}
\usepackage[T1, T2A]{fontenc}
\usepackage[utf8]{inputenc}
\usepackage[english, russian]{babel}
\usepackage{graphics}
\usepackage{enumitem}

\author{Иван Генрихович Эрлих}
\title{Основы вероятностей и теория меры}

\begin{document}
  \maketitle
  
  \paragraph{Семинарист:} Савелов Максим Павлович
  
  \section*{Предмет теории вероятностей}
  Предметом теории вероятностей могут являтся события, которые
  
  \subsection*{Задача о динозавре}
  Какова вероятность встретить динозавра на улице? 
  \begin{flushright}
  Записки невесты программиста (Алекс Экслер)
  \end{flushright}
    
  \subsection*{Физика на ПМИ}
  Вы первые два курса активно занимаетесь чем? Вы пишете код и занимаетесь фундаментальной математикой. Но изначально математика была фундаментом для естественных наук. Но теперь есть программирование. Идея физики на ПМИ в том, чтобы показать, как можно применить математический аппарат к жизни. В теорвере примерно то же самое. Вам формулируют жизненную задачу. Вы должны построить математическую модель эксперимента.
  
  \subsection*{Математическая модель эксперимента}
  Надо понять, будут ли применяться к этому эксперименту те методы, котрые применяются в теории вероятности. Рассмотрим две постановки задачи.
  
  \begin{itemize} 
  \item Какова вероятность, что монета упадет "орлом" вверх.
  \item Какова вероятность, что Вася Иванов получит по курсу ОВиТМ отл(9).
  \end{itemize}
  
  Эксперимент для первой постановки легко повторить. Второй эксперимент нельзя провести много раз: невозможно проверить адекватность ответов.
  
  Вместо этого сформулируем задачу так:
  \begin{itemize}
  \item Какова вероятность того, что студент со средним баллом 8,7 получит оценку отл(9).
  \end{itemize}
  
  \paragraph*{Требования на события:}
  \begin{enumerate}
  \item Возможность проводить неограничпнно большое число наблюдений.
  \item Статистическую устойчивось эксперимента.
  \end{enumerate}
  
  \paragraph{Статистическая устойчивость.} Пусть у нас есть несколько экспериментов: $ A_1, A_2 \ldots A_n $. Событие статистически устойчиво, если частоты некоторого события $ \mu(A_1), \mu(A_2) \ldots \mu(A_n) $ должны быть близки. 
  
  \paragraph*{Математическая модель эксперимента:}
  \begin{enumerate}
  \item Пространство элементарных исходов - $ \Omega = \{\omega\} $. Кажодое $\omega$ обозначает один из элементарных исходов. 
  \item Вероятность элементарных исходов - $ P = p(\omega) $. Задача состоит в том, чтобы задать вероятность.
  \end{enumerate}
  
  \paragraph*{Пример 1.} 2 раза бросаем монетку. Получаем 4 элементарных исхода: 
  \begin{enumerate}
  \item $ \omega_1 = (\text{о}, \text{о}),~ p(\omega_1) = \frac{1}{4} $
  \item $ \omega_2 = (\text{р}, \text{о}),~ p(\omega_2) = \frac{1}{4} $
  \item $ \omega_3 = (\text{о}, \text{р}),~ p(\omega_3) = \frac{1}{4} $
  \item $ \omega_4 = (\text{р}, \text{р}),~ p(\omega_4) = \frac{1}{4} $
  \end{enumerate}
  
  \paragraph*{Пример 2.} 2 раза бросаем монетку. Получаем 3 элементарных исхода: 
  \begin{enumerate}
  \item $ \omega_1 = \{\text{о}, \text{о}\},~ p(\omega_1) = \frac{1}{4} $
  \item $ \omega_2 = \{\text{о}, \text{р}\},~ p(\omega_2) = \frac{1}{2} $
  \item $ \omega_3 = \{\text{р}, \text{р}\},~ p(\omega_3) = \frac{1}{4} $
  \end{enumerate}

  \subsubsection*{Стандартные математические модели:}
  \begin{enumerate}
  \item Классическая.  $(\Omega, P), |\Omega| = n < \infty,~ P(\omega_i) = const,~ i = 1 \ldots n $.
  	\begin{enumerate}[label*=\arabic*.]
  	\item Урновая схема: выбор с порядком без возвращения. Шаров в урне: $N$. $\omega = (i_1, \ldots i_k)$, где $i_j$ - номер шара $ i_j \neq i_m,~ j \neq m $. $ | \Omega |  = N \cdot \ldots \cdot (N - k + 1) $.
  	\item Урновая схема: выбор с порядком с возвращением. Шаров в урне: $N$. $\omega = (i_1, \ldots i_k)$, где $i_j$ - номер шара $ i_j \neq i_m,~ j \neq m $. $ | \Omega |  = N^k $.
  	\item Урновая схема: выбор без порядка без возвращения. Шаров в урне: $N$. $\omega = (i_1, \ldots i_k)$, где $i_j$ - номер шара, номера могут совпадать. $ | \Omega |  = C_n^k = \frac{N!}{k!(N - k)!} $.
  	\item Урновая схема: выбор без порядка c возвращением. Шаров в урне: $N$. $\omega = (i_1, \ldots i_k)$, где $i_j$ - номер шара, номера могут совпадать. $ | \Omega |  = C_{n + k - 1}^k $.
  	\end{enumerate}
  \end{enumerate}
  
  \subsubsection*{Вероятность}
  $$ P(A) = \frac{|A|}{|\Omega|} $$
  
\end{document}

