\documentclass{article}

%\usepackage[a4paper, total={6in, 8in}]{geometry}

\usepackage{amsmath}
\usepackage{amsfonts}
\usepackage{amssymb}
\usepackage[T1, T2A]{fontenc}
\usepackage[utf8]{inputenc}
\usepackage[english, russian]{babel}
\usepackage{graphics}
\usepackage{amsthm}

\newtheorem*{df}{Определение}
\newtheorem{teo}{Теорема}
\newtheorem{lem}{Лемма}
\newtheorem{prp}{Предложение}
\newtheorem{hyp}{Предположение}
\newtheorem{ass}{Утверждение}
\newtheorem*{cor}{Следствие}
\newtheorem*{ntc}{Замечание}
\newtheorem*{xmp}{Пример}

\renewcommand\qedsymbol{$\blacksquare$}

\newcommand{\lb}{\left(}
\newcommand{\rb}{\right)}

\author{Эрлих Иван Генрихович}
\title{Основы вероятности и теория меры}

\begin{document}
\maketitle

\subsection*{Формула умножения вероятностей}
\begin{xmp}
  Урна: 5 белых, 3 красных шара. Последовательны вынимаются три. Найти вероятность события, что выпадет белый-красный-белый именно в такой последовательности.
  
  $$ P\{(\text{б},\text{к},\text{б})\} = \frac{5}{8} \cdot \frac{3}{7} \cdot \frac{4}{6} $$
  
  $$ P(A_1 \cap \ldots \cap A_n) = P(A_1) \cdot P(A_2|A_1) \cdot P(A_3|A_1 \cap A_2) \cdot \ldots \cdot P(A_n | A_1 \cap \ldots \cap A_{n-1}) $$
  
  $$ \frac{P(A_1)}{1} \cdot \frac{P(A_1 \cup A_2)}{P(A_1)} \cdot \ldots \cdot \frac{P(A_1 \cap A_2 \cap \ldots \cap A_n)}{P(A_1 \cap A_2 \cap \ldots \cap A_{n - 1})} $$
\end{xmp}

\subsection*{Формула Байеса (Апостериорная вероятность)}
$$ P(B_k | A) = \frac{P(AB_k)}{P_A} = \frac{P(AB_k)P(B_k)}{\sum\limits_{i=1}^{n} P(A|B_i) P(B_i)} $$
Пример с учителем:
$$ P (\text{плохой вызван} | \text{дан ответ} ) = \frac{2/3 \cdot 1/10}{31/90} = \frac{6}{31} $$

\subsection*{Геометрическая вероятность (Вопросы задания вероятностей)}

  $ A \subseteq \mathbb{R}^n \text{ , для $\Omega$ определен объем, конечн., положит.}$
  
  \noindent $ \mathcal{F} - ? = \{ A : A \subseteq \Omega\}$
  
  \noindent $ P(A) = \frac{\mu(A)}{\mu(\Omega)} $
  
  \noindent $ \mu(\Omega) < \infty $

\begin{xmp}(Задача о встрече)
Есть два друга, договорились встретиться с 12 до 13. Каждый ждет 15 минут, после чего уходит, если друг не пришел. Найти вероятность того, что они встретятся.

Задача для двумерного пространства $n = 2$. Построим график. Нам подходят такие координаты $(u, v) \in [12, 13]$ и $ |u - v| \leqslant \frac{1}{4} $.

$ \mu(\Omega) = 1,~~ \mu(A) = 1 - 2 \cdot \frac{1}{2} \left(\frac{3}{4}\right)^2 = \frac{7}{16} $
\end{xmp}

$$ P(A) = \frac{\mu(A)}{\mu(\Omega)} $$

\subsubsection*{Проблемы:}
\begin{enumerate}
\item Аксиоматика Колмогорова.
\item Системы множеств.
\item Строить меру.
\end{enumerate}

\begin{df} 
$\mathbb{T} над $\sigma$-алгеброй, если
\begin{enumerate}
	\item $\OMega \in \mathbb{T}$
	\item $\forall A \in \mathbb{T}, \overline{A} \in \mathbb{T}$
	\item $\forall A, B \in \mathbb{T},~ A \cup B \in \mathbb{T}$
\end{df}
\begin{df}
Пара $(\omega, \mathbb{T})$ множества и $\sigma$-алгебры на нем над измерениями пространством.
\end{df}
\begin{df}
Вероятностью (вероятностной мерой) на измеримом пространстве $(\omega, \mathbb{T})$ над
$$P: \mathbb{T} \rightarrow [0, 1]$$
\begin{enumerate}
	\item $P(\Omega) = 1$
	\item (счетная аддитивность) $\forall A_n \in \mathbb{T}: A_i \cap A_j = \emptyset,~ i \neq j$
	$$ P (\cup \limits_{n=1}^{\infty}A_n) = \sum \limits_{n=1}^{\infty} P(A_n) $$
\end{enumerate}
\section{Независимость}
\begin{df} 
События $A$ и $B$ называются независимымиЮ если $P(A \cap B) = P(A) \cdot P(B)$
\end{df}
\begin{xmp} Схема Бернулли доказать независимость (первая монетка орлом) ||_ (вторая монетка решкой) $n \geqslant 2$

Решение: 
$$\omega = (i_1, \ldots i_n), ij \in {0,1}$$
$$P(\omega) = p^{\sum ij}q^{n - \sum ij},~ p+q = 1$$
$$P(A) = \sum \limits_{\omega : (i_1, \ldots i_n)} P(\omega) = \sum \limits_{\omega : (i_1, \ldots i_n)} p^{1 + \sum \limitsj  j ij}q^{n - \sum ij} = $$
$$ = p \sum \limits_{(i_2, \ldots i_n)} p^{\sum ij}q^{n - \sum ij}
\end{document}
