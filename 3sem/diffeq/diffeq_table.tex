\documentclass{article}

%\usepackage[a4paper, total={6in, 8in}]{geometry}

\usepackage{amsmath}
\usepackage{amsfonts}
\usepackage{amssymb}
\usepackage[T1, T2A]{fontenc}
\usepackage[utf8]{inputenc}
\usepackage[english, russian]{babel}
\usepackage{graphics}
\usepackage{amsthm}
\usepackage[a4paper, total={6in, 8in}]{geometry}

\geometry{
 a4paper,
 total={170mm,257mm},
 left=20mm,
 top=20mm,
 }

\newcommand{\pd}[2]{\frac{\partial {#1}}{\partial {#2}}}

\author{Валентинов Александр}

\begin{document}
\begin{table}
   \begin{center}
   \caption{Методы решения дифференциальных уравнений}
   \begin{tabular}{|p{2.5cm}|p{6cm}|p{4cm}|p{2.5cm}|}
   \hline 
   \multicolumn{1}{|c}{Название} & \multicolumn{1}{|c}{Вид} & \multicolumn{1}{|c}{Алгоритм решения} & \multicolumn{1}{|c|}{Сводится к} \\ \hline 
   Уравнение с разделяющимися переменными \textbf{(РП)} & $$ P(x, y)dx + Q(x, y)dy = 0 $$ & \multicolumn{2}{|p{7.5cm}|}{Умножить или разделить на такое выражение, чтобы получить уравенение, одна часть которого содержит только $dx$ и функцию от $x$, а вторая только $dy$ и функцию от $y$}\\ \hline 
   Однородное уравнение \textbf{(ОУ)} & $$ P(x, y)dx + Q(x, y)dy = 0 $$ & $$ y = xz $$ & РП \\ \hline 
   Линейное уравнение \textbf{(ЛУ)}& $$ y' + a(x)y = f(x) $$ & \multicolumn{2}{|p{7.5cm}|}{Найти общее решение соответствующего однородного уравнения $y' + a(x)y = 0$. Представить константу в этом решении как функцию от $x$, подставить в исходное уравнение и найти эту функцию.}\\ \hline 
   Уравнение Бернулли & $$y' + a(x)y = b(x)y^m$$ & $$z = y^{1-m}$$ & ЛУ \\ \hline 
   Уравнение Рикатти & $$y' + a(x)y^2 + b(x)y + c(x) = 0$$ & $y = z + y_0(x)$, где $y_0(x)$ - какое-нибудь решение & Уравнение Бернулли\\ \hline 
   Уравнение в полных дифференциалах & $$ P(x, y)dx + Q(x, y)dy = 0 = dU $$ & \multicolumn{2}{|p{7.5cm}|}{$$U(x, y) = C, ~~\pd{U}{x} = P(x, y),~~\pd{U}{y} = Q(x, y)$$} \\ \hline 
   Интегрирующий множитель & $$ P(x, y)dx + Q(x, y)dy = 0 $$ & Поделить или умножить на такое $f(x, y)$, чтобы уравнение стало уравнением в полных дифференциалах & Уравнение в полных дифференциалах\\ \hline 
   \multicolumn{4}{|c|}{Уравнения, допускающие понижения порядка} \\ \hline
   1 & $$ F(x, y^{(k)}, y^{(k+1)}, \ldots, y^{(n)}) = 0 $$ & \multicolumn{2}{|p{7.5cm}|}{$$ y^k = z $$} \\ \hline 
   2 & $$ F(y, y', y'', \ldots, y^{(n)}) = 0 $$ & \multicolumn{2}{|p{7.5cm}|}{$$ y' = p(y) $$} \\ \hline 
   3 & $$ F(x, y, y', \ldots, y^{(n)}) = 0 \Leftrightarrow $$ $$ \Leftrightarrow F(kx, ky, ky', \ldots, ky^{(n)}) = 0 $$ & \multicolumn{2}{|p{7.5cm}|}{$$ y' = yz $$} \\ \hline
   4 & $$ F(x, y, y', \ldots, y^{(n)}) = 0 \Leftrightarrow $$ $$ F(kx, k^m y, k^{m - 1} y', \ldots, k^{m - n}y^{(n)}) = 0 $$ & $$ x = e^t,~~ y = ze^{mt} $$ & 1 - 3 \\ \hline 
   \end{tabular}
   \end{center} 
   \end{table} 
\end{document} 