\documentclass{article}

%\usepackage[a4paper, total={6in, 8in}]{geometry}

\usepackage{amsmath}
\usepackage{amsfonts}
\usepackage{amssymb}
\usepackage[T1, T2A]{fontenc}
\usepackage[utf8]{inputenc}
\usepackage[english, russian]{babel}
\usepackage{graphics}
\usepackage{amsthm}
\usepackage{hyperref}
\hypersetup{
colorlinks,
citecolor=black,
filecolor=black,
linkcolor=black,
urlcolor=black
}

\newtheorem*{df}{Определение}
\newtheorem{teo}{Теорема}
\newtheorem{lem}{Лемма}
\newtheorem{prp}{Предложение}
\newtheorem{hyp}{Предположение}
\newtheorem{ass}{Утверждение}
\newtheorem*{cor}{Следствие}
\newtheorem*{ntc}{Замечание}
\newtheorem{xmp}{Пример}
\setcounter{secnumdepth}{0}

\renewcommand\qedsymbol{$\blacksquare$}
\newcommand{\pd}[2]{\frac{\partial {#1}}{\partial {#2}}}
\newcommand{\re}[1]{(\ref{#1})}
\newcommand{\lb}{\left(}
\newcommand{\rb}{\right)}

\author{Свирщевский Сергей Ростиславович}
\title{Дифференциальные уравнения}

\begin{document}
  \begin{titlepage}
  \maketitle
  \begin{center}
  {\itshape\footnotesize Набор: Александр Валентинов}
  
  {\itshape\footnotesize Об ошибках писать: \url{https://vk.com/valentiay}}
  \end{center}
  \tableofcontents
  \vfill
  \end{titlepage}
  
    \section{Лекция 1}
  \subsection{Введение}
  \subsubsection{Основные понятия}
  \begin{df}
  Обыкновенным дифференциальным уравнением {(ОДУ)} $n$-го порядка называется уравнение вида:
  \begin{equation}
  \label{diffur}
  F(x, y, y', \ldots y^{(n)})
  \end{equation}
  где $x$ - независимая переменная, $y(x)$ - искомая функция, $ y', \ldots, y^{(n)} $ - ее производные, $F$ - заданная функция, определенная в области $\Omega \subseteq \mathbb{R}^{n+2}$. Порядок $n$ уравнения равен порядку старшей производной, входящей в уравнение. 
  \end{df}
  
  \begin{df}
  Функция $y = \varphi(x)$, определенная на некотором интервале $X = (\alpha, \beta)$, называется решением уравнения (\ref{diffur}), если 
  \begin{enumerate}
  \item $\varphi(x)$ $n$ раз дифференцируемо на $X$,
  \item $x, \varphi(x), \varphi'(x), \ldots \varphi^{(n)} \in \Omega,~~ \forall x \in X$,
  \item $F(x, \varphi(x), \varphi'(x), \ldots, \varphi^{(n)}(x)) = 0$.
  \end{enumerate}
  \end{df}
  \begin{ntc}
  В этом определении в качестве $X$ можно взять полуинтервал или отрезок, т.е любой промежуток действительной оси.
  \end{ntc}
  \begin{ntc}
  Решение может быть определено на сколько можно малом интервале. Разные решения могут быть определены на разных интервалах.
  \end{ntc}
  
  \begin{df}
  Уравнение 
  \begin{equation}
  \label{normaldiffur}
  y^{(n)} = f(x, y, y', \ldots y^{(n-1)})
  \end{equation}
  где $f$ - заданная функция, определенная в некоторой области $D \subseteq \mathbb{R}^{n+1}$, называется разрешенным относительно старшей производной или уравнение в нормальной форме.
  \end{df}
  
  \begin{df}
  Функция $y = \varphi(x)$, определенная на некотором интервале $X = (\alpha, \beta)$, называется решением уравнения (\ref{normaldiffur}), если 
  \begin{enumerate}
  \item $\varphi(x)$ $n$ раз дифференцируемо на $X$,
  \item $x, \varphi(x), \varphi'(x), \ldots \varphi^{(n - 1)} \in D,~~ \forall x \in X$,
  \item $\varphi^{(n)} \equiv f(x, \varphi(x), \varphi'(x), \ldots, \varphi^{(n - 1)}(x))$.
  \end{enumerate}
  \end{df}
  \begin{ntc}
  Процесс нахождения решений уравнения называется его интегрированием.
  \end{ntc}
  
  \begin{xmp}
  ~
  \begin{enumerate}
  \item $$ y' = f(x) \Rightarrow y = \int f(x) dx $$
  $$ y' = e^{-x^2},~~ y = \frac{1}{2}e^{-x^2} + C $$
  \item $$ y^{(n)} = f(x) \Rightarrow y = \int f(x) dx $$
  $$ y'' = 2x,~~ y' = x^2 + C_1,~~ y = \frac{1}{3}x^3 + C_1 x + C_2 $$ 
  \item $$ y' = ky,~~ k = const \neq 0 $$
  $$ y = Ce^{kx} $$
  \item $$ y' = y^2,~~ y = -\frac{1}{x} $$
  \end{enumerate}
  \end{xmp}

  \begin{ntc}
  Формулы, описывающие все решения уравнения содержат $n$ произвольных постоянных.
  \end{ntc}  
  
  \subsection{Задача Коши и теорема существования и единственности для уравнения (\ref{normaldiffur})}
  Для получения из множества решений какого-либо частного решения, необходимо задать дополнительные условия. Рассмотрим, например, уравнение первого порядка:
  \begin{equation}
  \label{lindiff}
     y' = f(x, y),~~ (x,y) \in D
  \end{equation}
  \begin{equation}
  \label{conditkoshi}
  y_0 = y(x_0)
  \end{equation}
  Возьмем точку $(x_0, y_0) \in D$ и рассмотрим начальное условие (НУ).
  \begin{df}
  Найти решение уравнения (\ref{lindiff}), удовлетворяющее НУ (\ref{conditkoshi})
  \end{df}
  
  \begin{teo}
  \label{koshi}
  Пусть функция $f(x, y)$ и ее частная производная $\frac{\partial f(x,y)}{\partial y}$ непрерывны в области $D\subseteq \mathbb{R}^2$. Тогда, $\forall (x_0, y_0) \in D$:
  \begin{enumerate}
  \item Существует решение задачи Коши, определенное на некотором интервале $ X \ni x_0 $
  \item Если $y_1(x)$, $y_2(x)$ - какие-либо решения, то $y_1(x) \equiv y_2(x)$ на пересечении их интервалов определения.
  \end{enumerate}
  \end{teo}
  
  \begin{xmp}
  $$ y' = kx ,~~ k = const \neq 0 $$
  Решение: $y = Ce^{kx}, ~~ x \in (-\infty, +\infty)$
  Докажем, что других решений нет. Пусть $y = \varphi(x), x \in X$ - какое-либо решение. Возьмем произвольную $x_0 \in X$ и найдем $y_0 = \varphi(x_0)$. Теперь покажем, что в этом семействе есть решение с такими же начальными условиями. Рассмотрим $ y = C_0 e^{kx},~~ C_0 = y_0e^{kx_0} $.  Оба этих решения являются решениями одной и той же задачи Коши с НУ $(x_0, y_0)$. В силу единственности по теореме (\ref{koshi}) мы имеем $\varphi(x)  = C_0 e^{kx}$ на $X$.
  \end{xmp}
  \begin{ntc}
  Решение $y = \varphi (x),~~ x \in X$ называется сужением решения $y = C_0e^{kx},~~ x \in \mathbb{R}$, на интервал $X$. А решение $y = C_0e^{kx}$ называется продолжением решения $y = \varphi (x)$ на $\mathbb{R} $.
  \end{ntc}
  
  \begin{df}
  Пусть $ y = \varphi_1(x),~~ x \in X_1 $ и $ y = \varphi_2(x),~~ x \in X_2 $ - какие-либо решения уравнения, и пусть $ X_1 \subseteq X_2 $. Тогда решение $ y = \varphi_2(x) $ называется продолжением решения $ y = \varphi_1(x) $ на $X_2$.
  \end{df} 
  \begin{ntc}
  В дальнейшем докажем, что каждое решение может быть продолжено на некоторый максимальный интервал до непродолжаемого решения.
  \end{ntc}
  
  \begin{df}
  График решения $ y = \varphi(x) $ на плоскости $(x, y)$  называется его интегральной кривой. Если под интегральной кривой понимать непродолжаемое решение, то теорему (\ref{koshi}) можно переформулировать так:
  
  Через каждую точку $(x_0, y_0) \in D$ проходит единственная интегральная кривая уравнения (\ref{lindiff}). 
  \end{df}
  
  \begin{ntc}
  Мы можем нарисовать интегральные кривые, не решая уравнение, поскольку мы знаем, как направлена касательная в любой точке.
  Не каждое уравнение не имеет аналитическое решение, например: $ y' = x^2 + y^2 $. В качестве альтернативы можно нарисовать на плоскости $(x, y)$ изоклины и получить представления о том, как выглядят интегральные кривые.
  \end{ntc}
  
  \begin{ntc}
  Для \textbf{существования} решения задачи Коши достаточно непрерывности функции $f(x, y)$. Но решение может быть не единственным.
  \end{ntc}
  
  \begin{xmp}
  $$ y' = 3\sqrt[3]{y^2} $$
  $$ \text{Решения: } y = 0, y = (x - C)^3 $$ В каждой точки интегральной кривой $y = 0$ нарушается единственность решения задачи Коши. Решение $y = 0$ называется особым. 
  \end{xmp}
  \section{Лекция 2}
  \begin{df}
  Решение уравнения (\ref{lindiff}) и его интегральная кривая $l$ называются особыми, если в любой окрестности каждой точки кривой $l$ через эту точку проходит, касаясь $l$, по крайней мере одна интегральная кривая уравнения (\ref{lindiff}), отличная от $l$.
  \end{df}
  \begin{ntc}
  Условие про касание $l$ избыточно.
  \end{ntc}
  
  Аналогично рассмотрим уравнение (\ref{normaldiffur}) при $n \geqslant 1$.
  Возьмем точку $(x_0, y_0, y'_0, \ldots y_0^{(n-1)}) \in D$. Рассмотрим НУ:
  \begin{equation}
  \label{five}
  y(x_0) = y_0,~ y'(x_0) = y'_0,~ \ldots y^{(n - 1)}(x_0) = y^{(n - 1)}_0
  \end{equation}   
  Задача Коши: найти  решение уравнения (\ref{normaldiffur}) для НУ (\ref{five})
  \begin{teo}
  Пусть функция $f$ и ее частные производные $f_y, f_{y'}, \ldots f_{y^{(n + 1)}} $ непрерывны в $ D \subseteq \mathbb{R}^{(n + 1)}$. Тогда задача Коши (\ref{normaldiffur}), (\ref{five}) имеет решение на интервале $X \ni x_0$. Любые два решения задачи (\ref{normaldiffur}), (\ref{five}) совпадают на пересечении их интервалов определения.
  \end{teo}
  
  \subsection{Уравнения первого порядка, интегрируемые в квадратурах}
  \subsubsection{Уравнения с разделяющимися переменными}
  $\boxed{y' = f(x)g(y) }$, где $f$ и $g$ - неперывны на $X$ и $Y$. 
  
  \noindent Схема решения:
  \begin{enumerate}
  
  \item $g(y) = 0 \Rightarrow$ постоянные решения $y = y_1$, $y = y_2$, $\ldots$
  \item $g(y) \neq 0$ На каждом интервале это выполнено.
  $$ \int \frac{y'(x)}{g(y(x))}dx = \int f(x)dx $$
  $$ \int \frac{dy}{g(y)} = \int f(x)dx $$
  $$ G(y) = F(x) + C \text{ - решение в неявной форме} $$
  $G$, $F$ - первообразные, $C$ - константа.
  Т.к. $G'(y) = \frac{1}{g(y)}$ на рассмаотриваемом интервале сохраняет знак, то $G(y)$ строго монотонна и, следовательно, имеет обратную. Поэтому можно написать явную формулу для решения.
  \end{enumerate}
  
  \subsubsection{Однородные уравнения}
  $\boxed{ y' = f\left(\frac{y}{x}\right) }$
  
  \noindent Замена: $\frac{y(x)}{x} = z(x)$, $y = xz$. 
  
  \noindent $ xz' + z = f(z) $, $xz' = f(z) - z $ - уравнение, с разделяющимися переменными.
  
  \begin{ntc}
  Уравнение инвариантно относительно растяжения: $ x \rightarrow ax $, $ y \rightarrow ay $; $ a > 0 $.
  \end{ntc}
  
  \subsubsection{Обобщенные однородные уравнения}
  $\boxed{ \frac{1}{x^{m - 1}}\frac{dy}{dx} = f \left(\frac{y}{x^m}\right) }$, $ x \neq 0 $
  
  \noindent Замена $\frac{y}{x^m} = z$, $y = x^m z$.
  
  \subsubsection{Линейные уравнения}
  $\boxed{ y' + a(x)y = b(x) }$
  
  \noindent Схема решения:
  \begin{enumerate}
  \item Рассматриваем однородное уравнение $y' + a(x)y = 0$, $y' = -a(x)y$,
    \begin{enumerate}
    \item $y = 0$ - решение.
    \item $y \neq 0 \Rightarrow$ $\int \frac{dy}{y} -\int a(x)dx$.
    $$ ln|y| = A(x) + \dot{C},~~ A(x) = \int a(t)dt $$
    $$ y = Ce^{-A(x)} $$
    \end{enumerate}
    \item Ищем решение в виде $ y = C(x)e^{-A(x)} $.
    $$ C'(x)e^{-A(x)} + C(x)e^{-A(x)}\left(-a(x)\right) + C(x)e^{-A(x)}\left(a(x)\right) = b(x) $$
    $$ C(x) = \int\limits_{x_0}^{x} b(t)e^{A(t)}dt + C_0 $$
    Общее решение:
    $$ y = e^{-A(x)} \int\limits_{x_0}^{x}b(t)e^{A(t)}dt + C_0e^{-A(x)} $$
    Задача Коши с НУ $y(x_0) = y_0$, $C_0 = y_0$.
  \end{enumerate}
  Каждое решение линейного уравнения определено на всем интервале $X$.
  
  \subsubsection{Уравнение Бернулли}
  $ \boxed{y' + a(x)y = b(x) y^n} $, $n \neq 0, 1 $
  
  \noindent Схема решения.  
  При $n > 0$ имеется решение $y = 0$.
  Пусть $y \neq 0$. Разделим на $y^n$:
  $$ y^{-n}y' + a(x)y^{1 - n} = b(x) $$
  $$ \frac{1}{1 - n}(y^{1-n})' + a(x)y^{1 - n} = b(x) $$
  $$ \text{Замена: } y^{1 -n} = z $$
  
  \subsubsection{Уравнения Риккати}
  $ \boxed{y' = a(x)y^2 + b(x) y + c(x)} $ 
  
  \noindent В общем случае не решается в квадратурах. Рассмотрим случай, когда известно какое-либо частное решение $y_0(x)$. 
  
  \noindent Замена: $y = z + y_0(x)$
  
  $$ z' + y_0' = a(z^2 + 2zy_0 + y_0^2) + b(z + y_0) + C $$
  $$ z' = az^2 + (2ay_0 + b)z \text{ - уравнение Бернулли} $$
  
  \subsubsection{Уравнение в полных дифференциалах}
  \subparagraph*{Уравнения 1-го порядка в симметричной форме}  
  \begin{equation} 
  \label{fulldiff}
  \boxed{P(x, y)dx + Q(x,y)dy = 0}
  \end{equation} 
  где $P, Q$ непрерывны в области $D \subset \mathbb{R}^2$, $P^2 + Q^2 \neq 0$ в $D$. 
  
  Уравнение называется уравнением в полных дифференциалах, если существует функция $U(x, y)$ непрерывно дифференцируемая в области $D$, такая, что 
  \begin{equation}
  \label{six}
  dU = Pdx + Qdy
  \end{equation} 
  в $D$. $dU(x,y) = 0$, $U(x, y) = C$ - содержит все решения $x(y)$ и $y(x)$.
  
  Пусть выполнено (\ref{six}), тогда $ P = \frac{\partial U}{\partial x} $, $ Q = \frac{\partial U}{\partial y} $.
  Пусть $P_y$ и $Q_x$ непрерывны в $D$. Тогда имеем:
  $$ P_y = \frac{\partial^2 U}{\partial x \partial y} = \frac{\partial^2 U}{\partial y \partial x} = Q_x \Rightarrow $$
  $$ \Rightarrow \frac{\partial P}{\partial y} = \frac{\partial Q}{\partial x} $$ 
   - необходимое условие того, что (\ref{fulldiff}) - уравнение в полных дифференциалах.
  
  \begin{ntc}
  Если область $D$ односвязна, то это условие является и достаточным. (Из курса математического анализа известно, что любой замкнутый контур можно стянуть в точку в этой области)
  \end{ntc}
    \section{Лекция 3}
  \subsubsection{Уравнение в полных дифференциалах}
  \begin{equation}
  \label{three.five}
  \boxed{P(x, y)dx + Q(x,y)dy = 0}
  \end{equation} 
  где $P, Q$ непрерывны в области $D \subset \mathbb{R}^2$, $P^2 + Q^2 \neq 0$ в $D$. 
  
  Уравнение называется уравнением в полных дифференциалах, если существует функция $U(x, y)$ непрерывно дифференцируемая в области $D$, такая, что 
  \begin{equation}
    \label{three.seven}
    dU = Pdx + Qdy
  \end{equation}

  \begin{equation}
    \label{three.eight}
    P = \pd{U}{x}, ~~ Q = \pd{U}{y}
  \end{equation} 
  Пусть $D: a < x < b,~ c < y < d$. Пусть выполняется (\ref{three.eight}), пусть $(x_0, y_0) \in D$ - произв. Найдем $U$ из \re{three.seven}:

  \[ \pd{U}{x} = P(x, y) \Rightarrow U(x, y) = \int\limits_{x_0}^xP(\xi, y)d\xi + \varphi(y) \]
  \[ \pd{U}{y} = Q(x, y) \Rightarrow \int\limits_{x_0}^{x}\pd{P(\xi, y)}{y}d\xi + \varphi'(y) = Q(x, y) \]
  \[ \int\limits_{x_0}^{x}\pd{Q(\xi, y)}{\xi}d\xi + \varphi'(y) = Q(x, y) \]
  \[ Q(x, y) - Q(x_0, y) + \varphi'(y) = Q(x ,y) \]
  \[ \varphi'(y) = Q(x_0, y) \]
  \[ \varphi(y) = \int\limits_{y_0}^{y}Q(x_0, \eta)d\eta \]

  \subsubsection{Интегрирующий множитель}
  Пусть уравнение \re{three.five} не является уравнением в полных дифференциалах.
  \begin{df}
  Функция $\mu(x, y)$ непрерывна в области $D, \mu(x, y) \neq 0$ в $D$, называется интегрирующим множителем уравнения \re{three.five}, если после умножения на нее \re{three.five} становится уравнением в полных дифференциалах, т.е.

  \[ \mu Pdx + \mu Qdy = dU \]
  \end{df}

  Если $\mu \in C^1(D)$ и $P$ и $Q$ удовлетворяют введенным ранее условиям, то делим уравнение на $\mu$:
  \[ \pd{(\mu P)}{y} = \pd{(\mu Q)}{x} \text{ или } Q\pd{\mu}{x} - P\pd{\mu}{y} = \mu\left(\pd{P}{y} - \pd{Q}{x} \right) \]

  Это уравнение с частными производными, вообще говоря, более сложное, чем исходное. Иногда удается найти его частное решение. Например, пусть  $\mu = \mu(y)$, тогда при $P \neq 0$

  \[ \pd{\mu}{y} = \omega\mu \text{, где } \omega = \frac{1}{p} \left( \pd{Q}{x} - \pd{P}{y} \right)\]

  Если $\omega = \omega(y)$, то для $\mu$ имеем ОДУ.

  \subsection{Методы понижения порядка дифференциальных уравнений}
  Рассматриваем уравнения вида
  \begin{equation}
  \label{three.one}
  F(x, y, y', \ldots , y^{(n)}) = 0
  \end{equation}
  допускающее понижение порядка с помощью некоторой замены переменных.
  
  \subsubsection{Уравнения, не содержащие $y , y', \ldots, y^{(k-1)}$}
  \begin{equation}
  \label{three.two}
  F(x, y^{(k)}, \ldots, y^{(n)}) = 0,~~ 1 \leqslant k \leqslant n
  \end{equation}
  Замена $z = y^{(k)}$. Получаем уравнение
  \[ F(x, z, \ldots, z^{(n - k)}) = 0 \]
  $(n - k)$-го порядка.

  \subsubsection{Уравнения, не содержащие $x$ (автономные)}
  \begin{equation}
  \label{three.three}
  F(y, y', \ldots, y^{(n)}) = 0
  \end{equation}
  \begin{enumerate}
  \item Находим решения вида $y = const$ (если такие есть)
  \item Пусть $y \neq const$. Замена $y' = z(y) = f_0(z)$

  \noindent $y'' = z'z = f_1(z, z'); ~~ y''' = (z'z)'z = f_2(z, z', z'')$

  \noindent $y^{(n)} = f_{n - 1}(z, z', \ldots, z^{(n-1)})$

  Получаем уравнение
  $$ \tilde{F}(y, z, z', \ldots, z^{(n -1)}) = 0 $$
  $(n - 1)$-го порядка.
  \end{enumerate}

  \subsubsection{Уравнения, однородные относительно $y, y', \ldots y^{(n)}$}
  Пусть $F$ удовлетворяет условию $\forall \lambda > 0$
  \begin{equation}
  \label{three.four}
  F(x , \lambda y, \lambda y', \ldots, \lambda y^{(n)}) = \lambda^m F(x , y, y', \ldots, y^{(n)})
  \end{equation}
  Замена $\frac{y'}{y} = z(x)$ при $y \neq 0$. Имеем:

  \[ y' = yz = yf_0(z);~~ y'' = yz' + y'z = y(z' + z^2) = yf_1(z, z'); \]
  \[\ldots\]
  \[y ^{(n)} yf_{n-1}(z, z', \ldots, z^{(n-1)}) \]

  \noindent Уравнение \re{three.one} примет вид

  \[ F(x, y, yf_0, \ldots, yf_{n - 1}) = 0 \]
  Пусть $y > 0$. В силу \re{three.four}
  \[ y^mF(x, 1, f_0, \ldots, f_{n-1}) = 0\]
  \[\tilde{F}(x, z, z', \ldots, z^{(n-1)}) = 0 \text{ --- уравнение $(n-1)$-го порядка}\]
  \begin{xmp}{$y'' = \sqrt{y'^2 + y^2}$}
  \begin{enumerate}
  \item $y = 0$ --- решение.
  \item $y \neq 0 ~~~ y' = yz,~ y'' = y(z' + z^2)$

  \noindent $y(z' + z^2) = \sqrt{y^2(z^2 + 1)}$

  \noindent $y(z' + z^2) = \pm y\sqrt{z^2 + 1}$

  \noindent $z' = -z^2 \pm \sqrt{z^2 + 1}$
  \end{enumerate}
  \end{xmp}

  \subsubsection{Обобщенные однородные уравнения}
  Это уравнения \re{three.one} с функцией $F$, удовлетворяющей условию
  \begin{equation}
  \label{three.five}
  F(\lambda x, \lambda^ky, \lambda^{k - 1}y, \ldots, \lambda^{k - n}y^{(n)}) = \lambda^m F(x, y, y', \ldots, y^{(n)}),~~ \forall \lambda > 0
  \end{equation}
  В этом случае уравнение \re{three.one} инвариантно относительно растяжений $x \rightarrow \lambda x$, $y \rightarrow \lambda^k y$.

  \noindent Замена: $\frac{y}{x^k} = z(x)$, $x \neq 0$ Имеем:

  \[ y  =x^kz = x^kf_0(z);~ y' = x^{k-1}(xz' + kz) = x^{k - 1}f_1(z, z') \]
  \[ \ldots \]
  \[ y^{(n)} = x^{k - n} f_n(z, z', \ldots, z^{(n)}) \]

  \[ F(x, x^kf_0, x^{k-1}f_1, \ldots, x^{k-n}f_n) = 0 \]
  Пусть $x > 0$. Тогда 
  \[ x^mF(1, f_0, f_1, \ldots, f_n) = 0 \]
  \[ \tilde{F}(x, z, z', \ldots, z^{(n)}) = 0 \]
  Имеем:

  \[ y = x^kz = x^kf_0(z) \]
  \[ y' = x^{k-1}(xz' + kz) = x^{k-1}f_1(z, xz') \]
  \[ y'' = x^{k-2}f_2(z, xz', x^2z'') \]
  \[ \ldots \]
  \[ y^{(n)} = x^{k-n}f_n(z, xz', \ldots, x^nz^{(n)}) \]

  \[ x^mF(1, f_0, f_1, \ldots, f_n) = 0\]
  Группа: $x \rightarrow \lambda x$, $z \rightarrow z$

  \noindent Замена: $t = \ln x$, $x = e^t$. Уравнение примет вид:

  \[ \hat{F}(z, z', \ldots, z^{(n)}) = 0 \text{ --- автономное}\]
  \section{Лекция 4}
  \subsection{Уравнения первого порядка, не разрешаемые относительно производных}
  \subsubsection{Метод введения параметра}
  \begin{equation}
  \label{four.one}
  F(x, y, y') = 0
  \end{equation}
  $F$ определено и непрерывно в области $ \Omega \subseteq \mathbb{R}^3$. $X$ и $T$ - интервалы.
  \begin{df}
  $\varPhi: y = \varphi(x),~ x \in X$ называется решением уравнения \re{four.one}, если:
  \begin{enumerate}
  \item $\varphi(x)$ непрерывно дифференцируемо на $X$,
  \item $(x, \varphi(x), \varphi'(x)) \in \Omega ~~ \forall x \in X$,
  \item $F(x, \varphi(x), \varphi'(x)) \equiv 0$ на $x$.
  \end{enumerate}
  \end{df}

  \begin{df}
  Функции $x = \varphi(t)$, $y = \psi(t)~~ t \in T$ задают решение уравнения \re{four.one} в параметрической форме, если:
  \begin{enumerate}
  \item $\varphi(t)$ и $\psi(t)$ непрерывно дифференцируемы и $\varphi'(t) \neq 0$ на $T$,
  \item $\left(\varphi(t), \psi(t), \frac{\psi'(t)}{\varphi'(t)}\right) \in \Omega ~~ t \in T$,
  \item $F\left(\varphi(t), \psi(t), \frac{\psi'(t)}{\varphi'(t)}\right) \equiv 0$ на $T$.
  \end{enumerate}
  \end{df}

  Положим $y' = P$ и запишем \re{four.one} в эквивалентном виде:
  \begin{equation}
  \begin{cases}
  \label{four.two}
  F(x ,y, P) = 0 \\
  dy = Pdx \\
  \end{cases}
  \end{equation}

  Каждому решению $x = \varphi(t)$, $y = \psi(t)$ уравнения \re{four.one} соответствует решение $x = \varphi(t)$, $y = \psi(t)$, $P = \frac{\psi'(t)}{\varphi(t)}$ системы \re{four.two}. Каждому решению $x = \varphi(t)$, $y = \psi(t)$, $P = \chi(t)$, где $\chi(t) = \frac{\psi'(t)}{\varphi'(t)}$ соответствует решение $x = \varphi(t)$, $y = \psi(t)$ уравнения \re{four.one}.

  Будем предполагать, что уравнение
  \[ F(x, y, P) = 0 \]
  задает в $\mathbb{R}^3$ гладкую поверхность $S$, допускающую параметрическое представление
  \[ x = \varphi(u, v),~~ y = \psi(u, v),~~ P = \chi(u, v) \]
  где $(u, v) \in D,~~ D \subseteq \mathbb{R}^2$ --- некоторая область и выполняются условия:
  \begin{enumerate}
  \item $\varphi$, $\xi$ и $\chi$ непрерывно дифференцируемы в $D$, причем
  \[ 
  \begin{vmatrix}
  \varphi_u & \varphi_v \\
  \psi_u & \psi_v \\
  \end{vmatrix}^2
  +
  \begin{vmatrix}
  \psi_u & \psi_v \\
  \chi_u & \chi_v \\
  \end{vmatrix}^2
  + 
  \begin{vmatrix}
  \chi_u & \chi_v \\
  \varphi_u & \varphi_v \\
  \end{vmatrix}^2
  \neq 0 \text{ в $D$}\]
  \item Задано взаимооднозначное отображение $D \leftrightarrow S$
  \item $F(\varphi(u, v), \psi(u ,v), \chi(u, v) \equiv 0$ на $D$.
  \end{enumerate}

  В этом случае имеем

  \[ dy = Pdx \Rightarrow \psi_udu + \psi_vdv = \chi(u ,v)(\varphi_udu + \varphi_vdv) \]

  $ (\psi_u - \chi\varphi_u)du + (\psi_v - \chi\varphi_v)dv = 0$ --- уравнение в симметричной форме(разрешимо относительно производной).

  Если $v = g(u, C)$ --- его решение, то
  \[ x = \varphi(u, y(u, C)),~~ y = \varphi(u, y(u, C)) \]
  --- решение уравнения \re{four.one} в параметрической форме.

  \subsection{Задача Коши и теорема существования и единственности решения уравнения \re{four.one}}
  \[ F(x, y, y') = 0 \]
  $F$ определено и непрерывно в области $ \Omega \subseteq \mathbb{R}^3$.

  Возьмем точку $(x_0, y_0, p_0) \in \Omega$ такую, что
  \[ F(x_0, y_0, p_0) = 0 \]
  \textbf{Задача Коши}: найти решение уравнения \re{four.one}, удовлетворяющее условию 
  \begin{equation}
  \label{four.four}
  y(x_0) = y_0, y'(x_0) = p_0
  \end{equation}
  \begin{teo}{Существования и единственности решения уравнения \re{four.one}}
  Пусть функция $F$ непрерывно дифференцируема в области $\Omega$ и пусть 
  \begin{equation}
  \pd{F(x_0, y_0, p_0)}{p} \neq 0
  \end{equation}
  Тогда решение задачи \re{four.one}, \re{four.four} существует и единственно на некотором интервале $X \ni x_0$
  \end{teo}
  \begin{proof}
  В силу теоремы о неявной функции из уравнения $F(x, y, p) = 0$ можно выразить $p$:
  \[ p = f(x, y), \]
  \text{где $f$ непрерывно дифференцируема в некоторой окрестности $U(x_0, y_0)$}, единственна и удовлетворяет условию
  \[ f(x_0, y_0) = p_0 \]
  Получим задачу
  \begin{equation}
  \label{four.six}
  y' = f(x, y),~ y(x_0) = y_0
  \end{equation}
  По теореме \re{koshi} из лекции 1, существует единственное решение задачи \re{four.six}, определенное на некотором интервале $X \ni x_0$. Эта функция является решением уравнения \re{four.one}, удовлетворяющим Н.У. \re{four.four}.
  \end{proof}

  \begin{ntc}
  Если $\exists \tilde{p}_0 \neq p_0:~ F(x_0, y_0, \tilde{p}_0 = 0$, $\pd{F(x_0, y_0, \tilde{p}_0)}{p} \neq 0$, то получим другое решение $y = \tilde{\varphi}'(x)$, удовлетворяющее условиям
  \[ \tilde{\varphi}(x_0) = y_0,~~ \tilde{\varphi}'(x_0) = \tilde{p}_0 \]
  При этом единственность не нарушается, т.к. решение $y = \varphi(x)$ и $y = \tilde{\varphi}(x)$ соответсвуют разным начальным параметрам: $(x_0, y_0, p_0)$ и $(x_0, y_0, \tilde{p}_0$).
  \end{ntc}

  \subsubsection{Особые решения уравнения \re{four.one}}
  \begin{df}
  Решение $y = \varphi(x)$, $x \in X$ уравнения \re{four.one} и его интегральная кривая $l$ называются особыми, если в люблй окрестности каждоый точки кривой $l$ через эту точку проходит, касаясь $l$, по крайней мере одна интегральная кривая, отличная от $l$.
  \end{df}

  Теоремы единственности предыдущего пункта следует, что в особых точках кривой формально выполнены условия
  \begin{equation}
  \label{four.seven}
  F(x, y, p) = 0,~~ \pd{F(x, y, p)}{p} = 0
  \end{equation}

  Это необходимое условие особого решения.

  Исключим из \re{four.seven} $p$ (если это возможно), получим уравнение $\varPhi(x, y) = 0$, задающее на плоскости т.н. дискриминантную кривую (или дискриминантное множество). Дискриминантное множество содержит все особые решения, но может также включать и другие точки.

  \subsubsection{Алгоритм отыскания особых решений}
  Из \re{four.seven} находим дискриминантные кривые и проверяем:
  \begin{enumerate}
  \item Являются ли они решениями
  \item Если да, то касаются ли их в каждой точке другие решения
  \end{enumerate}
  Если в обоих случаях ответ положительный, то соответствующая ветвь дискриминантной кривой является особым решением.
  \section{Лекция 5}
\subsubsection{Уравнение Клеро}
\[ y = xy' - \frac{1}{4}y'^2 \]
\begin{enumerate}
\item 
\begin{flalign}
\label{five.star}
& y' = p \Rightarrow y = px - \frac{1}{4} p^2 &
\end{flalign}
\begin{flalign*}
& dy = pdx \Rightarrow pdx = pdx + xdp -\frac{1}{2}pdp \Rightarrow (x - \frac{1}{2}p)dp = 0 &
\end{flalign*}
\begin{enumerate}
\item $x = \frac{1}{2}p \Rightarrow p = 2x \xrightarrow{\re{five.star}} y = x^2$
\item $dp = 0 \Rightarrow P = C \xrightarrow{\re{five.star}} y = Cx - \frac{1}{4}C^2$
\end{enumerate}
\item Особые решения. Дискриминантное множество.
\begin{flalign*}
&
\begin{cases}
F = 0 \\
F'_p = 0 \\
\end{cases}
\Rightarrow
\begin{cases}
y = px - \frac{1}{4}p^2 \\
0 = x - \frac{1}{2}p \\
\end{cases}
\Rightarrow
y = x^2 \text{ --- дискриминантная кривая}
\end{flalign*}
Условия касания:
\begin{flalign*}
&
\begin{cases}
y_0(x) = y(x) \\
y_0'(x) = y'(x) \\
\end{cases}
\Rightarrow
\begin{cases}
x^2 = Cx - \frac{1}{4}C^2 \\
2x = C \\
\end{cases}
\Rightarrow
C = 2x
\Rightarrow &\\
& \Rightarrow C = 2x \Rightarrow \text{ есть касание с соответствующей кривой семейства (b) } &\\
\end{flalign*}
при $x = 0$ имеем $C = 0 \Rightarrow y = 0$ \\
при $x = 1$ $C = 2$, $y = 2x - 1$ 
\end{enumerate}

Еще одно уравнение:
\[ y'^3 - 27y^2 = 0 \Leftrightarrow y' = 3\sqrt[3]{y^2} \]
\begin{enumerate}
\item Рассмотрено на Л1. Решения:
\begin{flalign*}
& y = 0,~ y = (x - C)^3 &\\
\end{flalign*}
\item Особые решение:
\begin{flalign*}
& \begin{cases}
p^3 - 27y^2 = 0 \\
3p^2 = 0 \\
\end{cases}
\Rightarrow
p = 0,~~ y = 0 \text{ --- дискриминантная кривая} &\\
\end{flalign*}
\begin{enumerate}
\item является решением
\item Условие касания:
\[ \begin{cases}
0 = (x -C)^3 \\
0 = 3(x - C)^2 \\
\end{cases} 
\Rightarrow
C = x \Rightarrow \text{ есть касания}\]
\end{enumerate}
\end{enumerate}
\subsection{Линейные уравнения с переменными коэффициентами}
\subsubsection{Определение. Теорема о существовании и единственности}
\begin{equation}
\label{five.one}
L[y] \equiv y^{(n)} + a_1(t)y^{(n - 1)} + \ldots + a_{n - 1}(t)y' + a_n(t)y = f(t)
\end{equation}

$t \in T = (a, b)$, $a_i(t)$, $f(t)$ - заданные функции, определенные и непрерывные на $T$.

Начальные условия: 
\begin{equation}
\label{five.two}
y(t_0) = y_0,~~ y'(t_0) = y_0',~~ \ldots,~~ y^{(n - 1)}(t_0) = y_0^{(n - 1)}
\end{equation}
где $t_0 \in T,~~ y_0,~~ y_0',~~ \ldots,~~ y_0^{(n - 1)} \in \mathbb{R}$ --- заданные числа

\paragraph*{Задача Коши:} найти решение уравнения \re{five.one}, удовлетворяющее условию~\re{five.two}

\begin{teo}{О существовании и единственности}
$\forall t_0 \in T,~~ y_0,~y_0',~ \ldots,~y_0^{(n - 1)} \in \mathbb{R}$ существует единственное решение задачи Коши \re{five.one}, \re{five.two}, определенное на \emph{всем} интервале $T$.
\end{teo}
\begin{ntc}
Если $a_i(t),~f(t)$ --- постоянные, то решение задачи Коши \re{five.one}, \re{five.two}, определенно на $\mathbb{R}$. 

Если в \re{five.one} $f(t) \neq 0$, то уравнение называется \emph{неоднородным}.

\label{five.three}
Уравнение $L[y] = 0$ называется \emph{однородным} (соответствующим уравнению \re{five.one})
\end{ntc}

\subsubsection{Линейность оператора $L$ и ее следствия}
\subparagraph{Линейность}
\begin{lem}{Линейность}
Оператор $L$ является линейным, т.е. $\forall$ различных непрерывно дифференцируемых функций $u(t), v(t), t \in T$ и $\forall$ чисел $A$ и $B$.

\[ L[Au(t) + Bv(t)] = AL[u(t)] + BL[v(t)] \]
\end{lem}
\begin{proof}
\begin{flalign*}
& L[Au + Bv] = (Au + Bv)^{(n)} + a_1(Au + Bu)^{(n - 1)} + \ldots + a_n(Au + Bv) = &\\
& = A(u^{(n)} + a_1u^{(n - 1)} + \ldots + a_nu) + B(v^{(n)} + a_1v^{(n - 1)} + \ldots + a_n v) = &\\
& = AL[u(t)] + BL[v(t)] &\\
\end{flalign*}
\end{proof}
\subparagraph{Структура общего решения уравнения \re{five.one}}
\begin{df}
Общим решением линейного уравнения \re{five.one} наызвается множество функций, содержащих все решения уравнения и только их.
\end{df}

В дальнейшем для однородного уравнения получим формулу общего решения, содержащее $n$ произвольных постоянных.
\begin{teo}
Общее решение уравнения \re{five.one} представляется в виде 
\begin{equation}
\label{five.four}
y(t) = y_0(t) + y_r(t)
\end{equation}
где $y_0(t)$ - общее решение однородного уравнения, $y_r(t)$ - произвольное частное решение \re{five.one}.
\end{teo}
\begin{proof}
Пусть $y_r(t)$ - какое-либо частное решение уравнения \re{five.one}. Замена:
\[ y(t) = z(t) + y_r(t) \]
Получим: $L[z] + L[y_r] = f(t)$, т.е. $L[z] = 0$. Это однородное уравнение. Его общее решение $z = y_0(t) \Rightarrow$ общее решение уравнения \re{five.one} имеет вид $y(t) = y_0(t) + y_r(t)$
\end{proof}
\subparagraph{Принцип суперпозиции}
\begin{lem}
Если $y_1(t), \ldots, y_k(t)$ --- решения однородного уравнения, то
\[ y(t) = C_1y_1(t) + \ldots + C_ky_k(t), \]
где $C_1, \ldots, C_k$ - произвольные постоянные, также являющиеся решениями однородного уравнения
\end{lem}
\begin{proof}
$L[y] = C_1\underbrace{L[y_1]}_0 + \ldots + C_k\underbrace{L[y_k]}_0 = 0$
\end{proof}

\begin{ntc}
Множество решений однородного уравнения является линейным пространством. Далее докажем, что его размерность $n$
\end{ntc}
\begin{lem}
Пусть в \re{five.one} $f(t) \sum\limits_{i = 1}^{k}f_i(t)$, и пусть $y_i(t)$ --- решение уравнений $L[y] = f_i(t),~~ i = 1, \ldots, k$, тогда функция $y(t) = \sum\limits_{i = 1}^{k}y_i(t)$ является решением уравнения \re{five.one}
\end{lem}
\begin{proof}
$L[\sum\limits_{i = 1}^k y_i] = \sum\limits_{i = 1}^kL[y_i] = \sum\limits_{i = 1}^k f_i(t) = f(t)$
\end{proof}
  \section{Лекция 6}
\subsubsection{Линейные однородные уравнения}
\begin{equation}
\label{six.three}
L[y] = 0
\end{equation}
\subparagraph{Линейно зависимые/независимые функции}
\begin{df}
Функции $y_1(t),~ \ldots,~ y_k(t),~~ t \in T$ называются линейно зависимыми, если существуют числа $C_1,~ \ldots,~ C_k$, $|C_1| + \ldots + |C_k| \neq 0$, такие что 
\begin{equation}
\label{six.four}
C_1y_1(t) + \ldots + C_ky_k(t) \equiv 0 \text{ на $T$}
\end{equation}
\end{df}
\begin{df}
Функции $y_1(t), \ldots, y_n(t)$ называются линейно независимыми, если \re{six.four} выполнено только при $C_1 = \ldots = C_k = 0$
\end{df}

Пусть $y_i(t),~~i = 1, \ldots, n$ имеет производную до $n - 1$ порядка $\forall t \in T$:
\[
\begin{cases}
C_1y_1(t) + \ldots + C_ky_k(t) = 0 \\
C_1y'_1(t) + \ldots + C_ky'_k(t) = 0 \\
C_1y^{k - 1}_1(t) + \ldots + C_ky^{k - 1}_k(t) = 0 \\
\end{cases}
\]
\begin{df}
Определитель
\[ W(t) = W(y_1(t), \ldots, y_k(t)) = 
\begin{vmatrix}
y_1(t) & \ldots & y_k(t) \\
y_1'(t) & \ldots & y_k'(t) \\
\vdots & \ddots & \vdots \\
y_1^{(k - 1)}(t) & \ldots & y_k^{(k - 1)}(t)
\end{vmatrix}\]
называется определитель Вронского функций $\{y_i(t)\}$ или их вронскиан.
\end{df}

\begin{lem}
Если функции $y_1(t), \ldots, y_k(t)$ линейно зависимы на $T$, то $W(y_1, \ldots, y_k(t))~\equiv~0$
\end{lem}

\subparagraph*{Линейная зависимость/независимость решений уравнения \re{six.three}}
Пусть $y_1(t), \ldots, y_n(t)$ --- решения уравнения \re{six.three}, а $W(T)$ --- их вронскиан.
\begin{lem}
Если $W(t_0) = 0,~~\forall t_0 \in T$, то решение $y_1(t), \ldots y_n(t)$ линейно зависимо на $T$.
\end{lem}
\begin{proof}
\begin{flalign*}
&
W(t_0) = 0 \Rightarrow Y_i = 
\left(
\begin{matrix}
y_i(t_0) \\
\vdots \\
y_i^{(n - 1)}(t_0) \\
\end{matrix}
\right)
\text{ --- линейно зависимые}
&\\
& \exists C_1, \ldots, C_n;~~ |C_1| + \ldots + |C_n| \neq 0 \rightarrow C_1Y_1 + \ldots + C_nY_n = 0 &\\
\end{flalign*}
Составим функцию:
\begin{flalign*}
& y(t) = C_1y_1(t) + \ldots + C_ny_n(t) &\\
& y(t) \text{ --- решение уравнения \re{six.three} и } y(t_0) = 0 \Rightarrow &\\
& \Rightarrow \text{ по теореме о существовании и единственности } y(t) \equiv 0 &\\
& \text{т.е. } C_1y_1(t) + \ldots + C_ny_n(t) \equiv 0 \text{ на } T \Rightarrow y_1, \ldots, y_n \text{ --- линейно зависимые}
\end{flalign*}
\end{proof}
\end{document} 

