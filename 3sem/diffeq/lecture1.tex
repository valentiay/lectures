  \section{Лекция 1}
  \subsection{Введение}
  \subsubsection{Основные понятия}
  \begin{df}
  Обыкновенным дифференциальным уравнением {(ОДУ)} $n$-го порядка называется уравнение вида:
  \begin{equation}
  \label{diffur}
  F(x, y, y', \ldots y^{(n)})
  \end{equation}
  где $x$ - независимая переменная, $y(x)$ - искомая функция, $ y', \ldots, y^{(n)} $ - ее производные, $F$ - заданная функция, определенная в области $\Omega \subseteq \mathbb{R}^{n+2}$. Порядок $n$ уравнения равен порядку старшей производной, входящей в уравнение. 
  \end{df}
  
  \begin{df}
  Функция $y = \varphi(x)$, определенная на некотором интервале $X = (\alpha, \beta)$, называется решением уравнения (\ref{diffur}), если 
  \begin{enumerate}
  \item $\varphi(x)$ $n$ раз дифференцируемо на $X$,
  \item $x, \varphi(x), \varphi'(x), \ldots \varphi^{(n)} \in \Omega,~~ \forall x \in X$,
  \item $F(x, \varphi(x), \varphi'(x), \ldots, \varphi^{(n)}(x)) = 0$.
  \end{enumerate}
  \end{df}
  \begin{ntc}
  В этом определении в качестве $X$ можно взять полуинтервал или отрезок, т.е любой промежуток действительной оси.
  \end{ntc}
  \begin{ntc}
  Решение может быть определено на сколько можно малом интервале. Разные решения могут быть определены на разных интервалах.
  \end{ntc}
  
  \begin{df}
  Уравнение 
  \begin{equation}
  \label{normaldiffur}
  y^{(n)} = f(x, y, y', \ldots y^{(n-1)})
  \end{equation}
  где $f$ - заданная функция, определенная в некоторой области $D \subseteq \mathbb{R}^{n+1}$, называется разрешенным относительно старшей производной или уравнение в нормальной форме.
  \end{df}
  
  \begin{df}
  Функция $y = \varphi(x)$, определенная на некотором интервале $X = (\alpha, \beta)$, называется решением уравнения (\ref{normaldiffur}), если 
  \begin{enumerate}
  \item $\varphi(x)$ $n$ раз дифференцируемо на $X$,
  \item $x, \varphi(x), \varphi'(x), \ldots \varphi^{(n - 1)} \in D,~~ \forall x \in X$,
  \item $\varphi^{(n)} \equiv f(x, \varphi(x), \varphi'(x), \ldots, \varphi^{(n - 1)}(x))$.
  \end{enumerate}
  \end{df}
  \begin{ntc}
  Процесс нахождения решений уравнения называется его интегрированием.
  \end{ntc}
  
  \begin{xmp}
  ~
  \begin{enumerate}
  \item $$ y' = f(x) \Rightarrow y = \int f(x) dx $$
  $$ y' = e^{-x^2},~~ y = \frac{1}{2}e^{-x^2} + C $$
  \item $$ y^{(n)} = f(x) \Rightarrow y = \int f(x) dx $$
  $$ y'' = 2x,~~ y' = x^2 + C_1,~~ y = \frac{1}{3}x^3 + C_1 x + C_2 $$ 
  \item $$ y' = ky,~~ k = const \neq 0 $$
  $$ y = Ce^{kx} $$
  \item $$ y' = y^2,~~ y = -\frac{1}{x} $$
  \end{enumerate}
  \end{xmp}

  \begin{ntc}
  Формулы, описывающие все решения уравнения содержат $n$ произвольных постоянных.
  \end{ntc}  
  
  \subsection{Задача Коши и теорема существования и единственности для уравнения (\ref{normaldiffur})}
  Для получения из множества решений какого-либо частного решения, необходимо задать дополнительные условия. Рассмотрим, например, уравнение первого порядка:
  \begin{equation}
  \label{lindiff}
     y' = f(x, y),~~ (x,y) \in D
  \end{equation}
  \begin{equation}
  \label{conditkoshi}
  y_0 = y(x_0)
  \end{equation}
  Возьмем точку $(x_0, y_0) \in D$ и рассмотрим начальное условие (НУ).
  \begin{df}
  Найти решение уравнения (\ref{lindiff}), удовлетворяющее НУ (\ref{conditkoshi})
  \end{df}
  
  \begin{teo}
  \label{koshi}
  Пусть функция $f(x, y)$ и ее частная производная $\frac{\partial f(x,y)}{\partial y}$ непрерывны в области $D\subseteq \mathbb{R}^2$. Тогда, $\forall (x_0, y_0) \in D$:
  \begin{enumerate}
  \item Существует решение задачи Коши, определенное на некотором интервале $ X \ni x_0 $
  \item Если $y_1(x)$, $y_2(x)$ - какие-либо решения, то $y_1(x) \equiv y_2(x)$ на пересечении их интервалов определения.
  \end{enumerate}
  \end{teo}
  
  \begin{xmp}
  $$ y' = kx ,~~ k = const \neq 0 $$
  Решение: $y = Ce^{kx}, ~~ x \in (-\infty, +\infty)$
  Докажем, что других решений нет. Пусть $y = \varphi(x), x \in X$ - какое-либо решение. Возьмем произвольную $x_0 \in X$ и найдем $y_0 = \varphi(x_0)$. Теперь покажем, что в этом семействе есть решение с такими же начальными условиями. Рассмотрим $ y = C_0 e^{kx},~~ C_0 = y_0e^{kx_0} $.  Оба этих решения являются решениями одной и той же задачи Коши с НУ $(x_0, y_0)$. В силу единственности по теореме (\ref{koshi}) мы имеем $\varphi(x)  = C_0 e^{kx}$ на $X$.
  \end{xmp}
  \begin{ntc}
  Решение $y = \varphi (x),~~ x \in X$ называется сужением решения $y = C_0e^{kx},~~ x \in \mathbb{R}$, на интервал $X$. А решение $y = C_0e^{kx}$ называется продолжением решения $y = \varphi (x)$ на $\mathbb{R} $.
  \end{ntc}
  
  \begin{df}
  Пусть $ y = \varphi_1(x),~~ x \in X_1 $ и $ y = \varphi_2(x),~~ x \in X_2 $ - какие-либо решения уравнения, и пусть $ X_1 \subseteq X_2 $. Тогда решение $ y = \varphi_2(x) $ называется продолжением решения $ y = \varphi_1(x) $ на $X_2$.
  \end{df} 
  \begin{ntc}
  В дальнейшем докажем, что каждое решение может быть продолжено на некоторый максимальный интервал до непродолжаемого решения.
  \end{ntc}
  
  \begin{df}
  График решения $ y = \varphi(x) $ на плоскости $(x, y)$  называется его интегральной кривой. Если под интегральной кривой понимать непродолжаемое решение, то теорему (\ref{koshi}) можно переформулировать так:
  
  Через каждую точку $(x_0, y_0) \in D$ проходит единственная интегральная кривая уравнения (\ref{lindiff}). 
  \end{df}
  
  \begin{ntc}
  Мы можем нарисовать интегральные кривые, не решая уравнение, поскольку мы знаем, как направлена касательная в любой точке.
  Не каждое уравнение не имеет аналитическое решение, например: $ y' = x^2 + y^2 $. В качестве альтернативы можно нарисовать на плоскости $(x, y)$ изоклины и получить представления о том, как выглядят интегральные кривые.
  \end{ntc}
  
  \begin{ntc}
  Для \textbf{существования} решения задачи Коши достаточно непрерывности функции $f(x, y)$. Но решение может быть не единственным.
  \end{ntc}
  
  \begin{xmp}
  $$ y' = 3\sqrt[3]{y^2} $$
  $$ \text{Решения: } y = 0, y = (x - C)^3 $$ В каждой точки интегральной кривой $y = 0$ нарушается единственность решения задачи Коши. Решение $y = 0$ называется особым. 
  \end{xmp}