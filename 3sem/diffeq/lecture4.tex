\section{Лекция 4}
  \subsection{Уравнения первого порядка, не разрешаемые относительно производных}
  \subsubsection{Метод введения параметра}
  \begin{equation}
  \label{four.one}
  F(x, y, y') = 0
  \end{equation}
  $F$ определено и непрерывно в области $ \Omega \subseteq \mathbb{R}^3$. $X$ и $T$ - интервалы.
  \begin{df}
  $\varPhi: y = \varphi(x),~ x \in X$ называется решением уравнения \re{four.one}, если:
  \begin{enumerate}
  \item $\varphi(x)$ непрерывно дифференцируемо на $X$,
  \item $(x, \varphi(x), \varphi'(x)) \in \Omega ~~ \forall x \in X$,
  \item $F(x, \varphi(x), \varphi'(x)) \equiv 0$ на $x$.
  \end{enumerate}
  \end{df}

  \begin{df}
  Функции $x = \varphi(t)$, $y = \psi(t)~~ t \in T$ задают решение уравнения \re{four.one} в параметрической форме, если:
  \begin{enumerate}
  \item $\varphi(t)$ и $\psi(t)$ непрерывно дифференцируемы и $\varphi'(t) \neq 0$ на $T$,
  \item $\left(\varphi(t), \psi(t), \frac{\psi'(t)}{\varphi'(t)}\right) \in \Omega ~~ t \in T$,
  \item $F\left(\varphi(t), \psi(t), \frac{\psi'(t)}{\varphi'(t)}\right) \equiv 0$ на $T$.
  \end{enumerate}
  \end{df}

  Положим $y' = P$ и запишем \re{four.one} в эквивалентном виде:
  \begin{equation}
  \begin{cases}
  \label{four.two}
  F(x ,y, P) = 0 \\
  dy = Pdx \\
  \end{cases}
  \end{equation}

  Каждому решению $x = \varphi(t)$, $y = \psi(t)$ уравнения \re{four.one} соответствует решение $x = \varphi(t)$, $y = \psi(t)$, $P = \frac{\psi'(t)}{\varphi(t)}$ системы \re{four.two}. Каждому решению $x = \varphi(t)$, $y = \psi(t)$, $P = \chi(t)$, где $\chi(t) = \frac{\psi'(t)}{\varphi'(t)}$ соответствует решение $x = \varphi(t)$, $y = \psi(t)$ уравнения \re{four.one}.

  Будем предполагать, что уравнение
  \[ F(x, y, P) = 0 \]
  задает в $\mathbb{R}^3$ гладкую поверхность $S$, допускающую параметрическое представление
  \[ x = \varphi(u, v),~~ y = \psi(u, v),~~ P = \chi(u, v) \]
  где $(u, v) \in D,~~ D \subseteq \mathbb{R}^2$ --- некоторая область и выполняются условия:
  \begin{enumerate}
  \item $\varphi$, $\xi$ и $\chi$ непрерывно дифференцируемы в $D$, причем
  \[ 
  \begin{vmatrix}
  \varphi_u & \varphi_v \\
  \psi_u & \psi_v \\
  \end{vmatrix}^2
  +
  \begin{vmatrix}
  \psi_u & \psi_v \\
  \chi_u & \chi_v \\
  \end{vmatrix}^2
  + 
  \begin{vmatrix}
  \chi_u & \chi_v \\
  \varphi_u & \varphi_v \\
  \end{vmatrix}^2
  \neq 0 \text{ в $D$}\]
  \item Задано взаимооднозначное отображение $D \leftrightarrow S$
  \item $F(\varphi(u, v), \psi(u ,v), \chi(u, v) \equiv 0$ на $D$.
  \end{enumerate}

  В этом случае имеем

  \[ dy = Pdx \Rightarrow \psi_udu + \psi_vdv = \chi(u ,v)(\varphi_udu + \varphi_vdv) \]

  $ (\psi_u - \chi\varphi_u)du + (\psi_v - \chi\varphi_v)dv = 0$ --- уравнение в симметричной форме(разрешимо относительно производной).

  Если $v = g(u, C)$ --- его решение, то
  \[ x = \varphi(u, y(u, C)),~~ y = \varphi(u, y(u, C)) \]
  --- решение уравнения \re{four.one} в параметрической форме.

  \subsection{Задача Коши и теорема существования и единственности решения уравнения \re{four.one}}
  \[ F(x, y, y') = 0 \]
  $F$ определено и непрерывно в области $ \Omega \subseteq \mathbb{R}^3$.

  Возьмем точку $(x_0, y_0, p_0) \in \Omega$ такую, что
  \[ F(x_0, y_0, p_0) = 0 \]
  \textbf{Задача Коши}: найти решение уравнения \re{four.one}, удовлетворяющее условию 
  \begin{equation}
  \label{four.four}
  y(x_0) = y_0, y'(x_0) = p_0
  \end{equation}
  \begin{teo}{Существования и единственности решения уравнения \re{four.one}}
  Пусть функция $F$ непрерывно дифференцируема в области $\Omega$ и пусть 
  \begin{equation}
  \pd{F(x_0, y_0, p_0)}{p} \neq 0
  \end{equation}
  Тогда решение задачи \re{four.one}, \re{four.four} существует и единственно на некотором интервале $X \ni x_0$
  \end{teo}
  \begin{proof}
  В силу теоремы о неявной функции из уравнения $F(x, y, p) = 0$ можно выразить $p$:
  \[ p = f(x, y), \]
  \text{где $f$ непрерывно дифференцируема в некоторой окрестности $U(x_0, y_0)$}, единственна и удовлетворяет условию
  \[ f(x_0, y_0) = p_0 \]
  Получим задачу
  \begin{equation}
  \label{four.six}
  y' = f(x, y),~ y(x_0) = y_0
  \end{equation}
  По теореме \re{koshi} из лекции 1, существует единственное решение задачи \re{four.six}, определенное на некотором интервале $X \ni x_0$. Эта функция является решением уравнения \re{four.one}, удовлетворяющим Н.У. \re{four.four}.
  \end{proof}

  \begin{ntc}
  Если $\exists \tilde{p}_0 \neq p_0:~ F(x_0, y_0, \tilde{p}_0 = 0$, $\pd{F(x_0, y_0, \tilde{p}_0)}{p} \neq 0$, то получим другое решение $y = \tilde{\varphi}'(x)$, удовлетворяющее условиям
  \[ \tilde{\varphi}(x_0) = y_0,~~ \tilde{\varphi}'(x_0) = \tilde{p}_0 \]
  При этом единственность не нарушается, т.к. решение $y = \varphi(x)$ и $y = \tilde{\varphi}(x)$ соответсвуют разным начальным параметрам: $(x_0, y_0, p_0)$ и $(x_0, y_0, \tilde{p}_0$).
  \end{ntc}

  \subsubsection{Особые решения уравнения \re{four.one}}
  \begin{df}
  Решение $y = \varphi(x)$, $x \in X$ уравнения \re{four.one} и его интегральная кривая $l$ называются особыми, если в люблй окрестности каждоый точки кривой $l$ через эту точку проходит, касаясь $l$, по крайней мере одна интегральная кривая, отличная от $l$.
  \end{df}

  Теоремы единственности предыдущего пункта следует, что в особых точках кривой формально выполнены условия
  \begin{equation}
  \label{four.seven}
  F(x, y, p) = 0,~~ \pd{F(x, y, p)}{p} = 0
  \end{equation}

  Это необходимое условие особого решения.

  Исключим из \re{four.seven} $p$ (если это возможно), получим уравнение $\varPhi(x, y) = 0$, задающее на плоскости т.н. дискриминантную кривую (или дискриминантное множество). Дискриминантное множество содержит все особые решения, но может также включать и другие точки.

  \subsubsection{Алгоритм отыскания особых решений}
  Из \re{four.seven} находим дискриминантные кривые и проверяем:
  \begin{enumerate}
  \item Являются ли они решениями
  \item Если да, то касаются ли их в каждой точке другие решения
  \end{enumerate}
  Если в обоих случаях ответ положительный, то соответствующая ветвь дискриминантной кривой является особым решением.