\documentclass{article}

%\usepackage[a4paper, total={6in, 8in}]{geometry}

\usepackage{amsmath}
\usepackage{amsfonts}
\usepackage{amssymb}
\usepackage[T1, T2A]{fontenc}
\usepackage[utf8]{inputenc}
\usepackage[english, russian]{babel}
\usepackage{graphics}
\usepackage{amsthm}

\newtheorem*{df}{Определение}
\newtheorem{teo}{Теорема}
\newtheorem{lem}{Лемма}
\newtheorem{prp}{Предложение}
\newtheorem{hyp}{Предположение}
\newtheorem{ass}{Утверждение}
\newtheorem*{cor}{Следствие}
\newtheorem*{ntc}{Замечание}

\renewcommand\qedsymbol{$\blacksquare$}

\newcommand{\lb}{\left(}
\newcommand{\rb}{\right)}

\author{Аланакян Юрий Робертович}
\title{Лекции по общей физике, электричество}

\begin{document}
    \maketitle
    
    \section*{Заряды.}
    
	Одноименные зараяды отталкиваются, разноименные - притягиваются.
	    
    \paragraph{Закон сохранения заряда.} Если система изолирована, какие бы процессы в ней не происходили, алгебраическая сумма зарядов остается постоянной.
    
    \paragraph{Закон Кулона.} Получен Кулоном благодаря эксперименту с крутильными весами.
    $$ \overrightarrow{F} = \frac{q_1q_2}{r^2} \frac{\overrightarrow{r}}{r} $$
    
    \paragraph{Напряженность электрического поля.} Напряженностью электрического поля называется сила, действующая на единичный точечный заряд.
   
    $$ \overrightarrow{E} = \frac{q}{r^2} \frac{\overrightarrow{r}}{r} $$
   
    \paragraph{Принцип суперпозиции.} Напряженности от различных зарядов складываются.
   
    \subsection*{Электрический диполь.} Электрическим диполем называется два одинаковых по абсолютной величине, но разноименных заряда, жестко соединенных между собой. Расстояние $\overrightarrow{l}$ между зарядами называеся плечом диполя. Плечо направлено от отрицательного заряда к положительному. Величина $\overrightarrow{p} = q\overrightarrow{l}$ называется дипольным моментом.
    
    \begin{center}
   
    \begin{picture}(100, 100) %
    \put(30, 50){\vector(1,0){40}} %
    \put(25, 50){\circle{10}} %
    \put(75, 50){\circle{10}} %
    \put(18, 63){$-q$} %
    \put(68, 63){$+q$} %
    \put(45, 30){$\overrightarrow{l}$} %
    \end{picture}
    
    \end{center}
	Напряженность электрического поля диполя:
    $$ \overrightarrow{E} = q \left( \frac{\overrightarrow{r}}{r^3} - \frac{\overrightarrow{r} + \overrightarrow{l}}{\vert \overrightarrow{r} + \overrightarrow{l}\vert^3} \right)$$
  
    $$ \overrightarrow{E} = \frac{3(\overrightarrow{p}\overrightarrow{r})\overrightarrow{r}}{r^3} - \frac{\overrightarrow{p}}{r^3} $$
    Поле вдоль диполя:
    $$ E_{\parallel} = q \frac{d}{dr}\left[\frac{1}{r^2} - \frac{1}{\left(r + l \right)^2} \right] = \frac{2\overrightarrow{p}}{r^3} $$
    Поле перпендикулярное диполю:
    $$ E_{\perp} = - \frac{\overrightarrow{p}}{r^3} $$

    \begin{center}
   
    \begin{picture}(150, 150) %
    \put(30, 50){\vector(1, 0){40}} %
    \put(25, 50){\circle{10}} %
    \put(75, 50){\circle{10}} %
    \put(18, 63){$-q$} %
    \put(68, 63){$+q$} %
    \put(45, 30){$\overrightarrow{l}$} %
    \put(100, 125){\circle{1}} %
    \put(102, 127){$A$} %
    \put(50, 75){\circle{10}} %
    \put(28, 53){\line(1, 1){18}} %
    \put(72, 53){\line(-1, 1){19}} %
    \put(54, 79){\vector(1, 1){46}} %
    \put(70, 115){$\overrightarrow{r}$} %
    \put(35, 85){$\pm q$} %
    \end{picture}
    
    \end{center}
        
    Возьмем произвольную точку $A$. Найдем поле в ней. Опускаем перендиклуяр на прямую, соединяющую диполь и точку $A$. В основании перпендикуляра поместим заряды $+q$ и $-q$, получим два диполя, эквивалентных первому, направденные праллельно и перпендикулярно прямой, соединяющей диполь и точку.
  
    $$ E_A = \frac{2\overrightarrow{p_1} - \overrightarrow{p_2}}{r^3} = \frac{3\overrightarrow{p_1} - \overrightarrow{p}}{r^3} = \frac{3(\overrightarrow{p}\overrightarrow{r})\overrightarrow{r}}{r^3} - \frac{\overrightarrow{p}}{r^3} $$
  
    \paragraph{Силовые линии.} Силовые линии - касательные к вектору $\overrightarrow{E}$.
  
    \paragraph{Однородное поле.} Силы, действующие на диполь: 
    $$ \overrightarrow{F_1} = q\overrightarrow{E} $$
    $$ \overrightarrow{F_2} = -\overrightarrow{F_1} = -q\overrightarrow{E} $$
    $$ \overrightarrow{M} = [\overrightarrow{p}\overrightarrow{E}] $$
  
    \paragraph{Неоднородное поле.}
    $$ \overrightarrow{F} = q\overrightarrow{E}(\overrightarrow{r}) - q\overrightarrow{E}(\overrightarrow{r} + \overrightarrow{l}) $$
  
    $$ \overrightarrow{E}= q \left(l_x \frac{\partial\overrightarrow{E}}{\partial x} + l_y \frac{\partial\overrightarrow{E}}{\partial y} + l_z \frac{\partial\overrightarrow{E}}{\partial z} \right)  = \left(\overrightarrow{p}\triangledown\right)\overrightarrow{E} $$ 

    \subsection*{Теорема Гаусса.}
    
    \subsubsection*{Поток вектора:}
    $$ d\Phi = \overrightarrow{A} d\overrightarrow{S} $$
  
    $$ \Phi = \int \overrightarrow{A}d\overrightarrow{S} $$
  
    $$ \Phi = \oint \overrightarrow{E}d\overrightarrow{S} $$    
    
    \subsubsection*{Дивергенция:}
    $$ div \overrightarrow{A} = \lim \limits_{V \rightarrow 0} \oint \frac{\overrightarrow{A}d\overrightarrow{S}}{V} $$
    $$ div \overrightarrow{E} = \triangledown \overrightarrow{E} $$ 
    
    \begin{teo}(Теорема гаусса в интегральной форме) Поток вектора напряженности в электрического поля в ограниенном объеме равен $ 4\pi q $.
  
    $$ q = \int \rho dV $$
  
    $$ \oint \frac{q}{r^3} \overrightarrow{r} d\overrightarrow{S} = q\int d\Omega $$
    
    $$ \boxed{\oint \overrightarrow{E} d\overrightarrow{S} = 4\pi q} $$
    \end{teo}
    \begin{teo}{(Теорема Гаусса в дифференциальной форме)}
    $$ \boxed{div \overrightarrow{E} = 4\pi \rho} $$
    \end{teo}
    
    % 2017-09-07    
    
    \subsection*{Теорема Остроградского-Гаусса}
    \begin{teo}
    $$ \int div \overrightarrow{A} dV = \oint \overrightarrow{A}d\overrightarrow{S} $$ 
    \end{teo}
    
    \subsection*{Теорема Ирншоу}
    \begin{teo}
    Невозможно создания статической конфигурации электрических зарядов. 
    \end{teo}
    \begin{proof}
    Следует из теоремы Гаусса.
    \end{proof}  
    \section*{Потенциал электрического поля}
    Пусть у нас имеется точечный заряд $Q$. Возьмем заряд $q$ и переместим из точки $1$ в точку $2$. Посчитаем работу:
    $$ A = \int\limits_1^2 \overrightarrow{E}d\overrightarrow{r} = qQ\int\limits_1^2 \frac{\overrightarrow{r}d\overrightarrow{r}}{r^3} = qQ\int\limits_1^2 \frac{dr}{r^2} = qQ\lb \frac{1}{r_1^2} - \frac{1}{r_2^2} \rb $$
    \begin{df}
    Разность потенциалов - работа, которую необходимо совершить  для перемещения единичного заряда ежду двумя точками. Потенциальным называется поле, работа в котором зависит тололько от начального и конечного положения.
    \end{df}
    $$ \overrightarrow{E} = -\triangledown \varphi $$
    $$ d\varphi = \overrightarrow{E}d\overrightarrow{l} $$
    
    Т.к. потенциальное поле скалярное, можно провести линии, на которых потенциал одинаков. Эти линии будут параллельны силовым линиям поля.
    
    \begin{ntc}
    Потенциалы нескольких зарядов суммируются. Это называется принципом суперпозиции.
    $$ \varphi = \int \frac{\rho(\overrightarrow{r})dV}{r} $$
    \end{ntc}
    
    \subsection*{Уравнение Пуассона}
    $$ \triangledown \triangledown \varphi = -4\pi\rho $$
    $$ \triangledown \triangledown = \triangle = \frac{\partial^2}{\partial x^2} + \frac{\partial^2}{\partial y^2} + \frac{\partial^2}{\partial z^2} \text{ - лапласиан} $$
    $$ \boxed{\triangle \varphi = - 4\pi\rho} $$
    
    \begin{ntc}
    Решения уравнения Лапласа называются гармоническими функциями.
    \end{ntc}
    
    \begin{teo} {(О циркуляции в электростатическом поле)}
    Циркуляция электростатического поля по замкнутому контуру равна нулю.
    $$ \oint \overrightarrow{E}d\overrightarrow{l} = 0 $$
    Такое поле называется безвихревым.
    \end{teo}
    
    \subsubsection*{Граничные условия}
    Есть плоская граница двух сред. 
    $$ E_n^{(1)} - E_n^{(2)} = 3\pi\sigma $$
    Тангенциальные составляющие равны между собой из теоремы о циркуляции.
    $$ E_t^{(1)} = E_t^{(2)} $$
    
    \paragraph{Следствие из теоремы Ирншоу.} Потенциал не может иметь максимума и минимума.
    \begin{teo}
    Уравнение Лапласа с граничными условиями имеет единственное решение.
    \end{teo}  
    \begin{proof}
    Пусть у нас есть несколько точек, в которых $\varphi_i = 0$.
    Предположим, что уравнение Лапласа имеет решения $\varphi$ и $\psi$. Т.к. уравнение линейное, $\varphi - \psi$ должно удовлетворять уравнению, но не граничным условиям, т.к. оно не имеет ни максимума, ни минимума и имеет ноль в нескольких точках.
    \end{proof}
    
    \subsubsection*{Ротор}
    $$ rot \overrightarrow{A} = \lim\limits_{s \rightarrow 0} \frac{\oint \overrightarrow{A}d\overrightarrow{l}}{S} $$
    
    \begin{ass}
    $$ rot \overrightarrow{A} = \triangledown \times \overrightarrow{A} $$
    \end{ass}
    \begin{proof}
    Возьмем в ДСК прямоугольный контур и вычислим по нему циркуляцию. 
    \end{proof}
\end{document}
