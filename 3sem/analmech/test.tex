\section{Основные теоремы динамики}
\subsection{Основные динамические величины}
\begin{df}
$ \v P = \sum\limits_{i = 1}^N m_i\v v_i $ --- импульс.
$ \v K_O = \sum\limits_{i = 1}^N[[\v r_i, m_i\v v_i] $ --- кинематический момент онисительно точки $O$.
$ K_l = (\v K_0, \v e_l)$ --- кинематический момент относительно оси $l$.
\end{df}
\begin{ntc}
$ O \in l,~~ \v e_l \parallel \v l$; $K_l$ не зависит от точки $O$.
\end{ntc}
\begin{df}
$ T = \frac{1}{2} = \sum m_i v_i^2 = \frac{1}{2}m_i(\v v_i, \v v_i)$ --- кинетическая энергия.
\end{df}
\begin{df}
$S$ --- центр масс системы: \[ \v r_s = \frac{\sum m_i \v r_i}{m} \]
\end{df}
\[ \v P = \sum m_i \frac{d \v r_i}{d\tau} = \frac{d}{dt}\lb \sum m_i \v r_i \rb = \frac{d}{dt}(m\v r_s) = m\v v_s \]
\[ \boxed{\v P = m\v v_s} \]
\begin{df}
Осями Кенига называется система отсчета с началом в центра масс системы и осями, параллельными неподвижным. (Движется поступательно вместе с цетром масс)
\end{df}
\[ \v r_i = \v R + \v \rho_i \]
\begin{df}
\[ \v K_{\text{кин}} = \sum [\v \rho_i, m \dot{\v \rho_i} \ldots \]
\end{df}

\begin{teo}[Формулы Кенига]
\[ \v K_O = [\v r_s, m \v v_s] + \v K_{\text{кен}} \]
\[ T = \frac{1}{2}m v_\rho^2 + T_{\text{кен}} \]
\end{teo}
\begin{proof}
\begin{flalign*}
& \v K_0 = \sum[\v R + \v \rho, m_i \dot{\v R} + m_i \dot{\v \rho}_i] = [\v R,  (\sum m_i)\dot{\v R}] + \left[\v R, \sum m_i\dot{\v \rho_i}\right] + &\\
& + \left[\sum m_i\v \rho_i, \dot {\v \rho}_i, \v R\right] + \sum[\v \rho_i m_i \dot{\v \rho}_i] = [\v r_s, m \v v_s] + \v K_{\text{кен}} &\\
\end{flalign*}
\end{proof}

\begin{teo}[Об изменении импульса]
\[ \dot{\v P} = \sum \v F_i^{(e)} = \v F \]
\end{teo}
\begin{proof}
\[ \dot{\v P}_i = \frac{d}{dt}\sum m_i\v v_i = \sum m_i \v w_i = \sum \v F_i^{(e)} + \underbrace{\sum \v F_i^{(i)}}_0 = \v F \]
\end{proof}

\begin{teo}[Формула движения центра масс]
\[ m \v w_s = \v F \]
\end{teo}
\begin{cor}
\[ \v F = 0 \Rightarrow \v w_s = 0 \Rightarrow \v v_s = \v v_0 = const \Rightarrow \v r_s = \v v_0(t - t_0) + \v r_0 \]
\end{cor}
\begin{cor}
\[ (\v F, \v e_x) = 0 \Rightarrow (\dot {\v P}, \v e_x) = 0 \Rightarrow \v v_x = const \]
\end{cor}
\begin{teo}[Теорема об изменении кинетического момента относительно неподвижного полюса]
\[ \v K_O = \sum [\v r_i, \v F_i^{(e)}] = \v M_O \]
\end{teo}
\begin{proof}
\begin{flalign*}
& \frac{d}{dt} \v K_O = \frac{d}{dt}\lb\sum [\v r_i, m_i\v v_i] \rb = \sum \left[\frac{d \v r_i}{dt}, m_i\v v_i\right] + \sum [\v r_i, m_i\dot{\v v}_i] = &\\
& = \sum[\v r_i, \v F_i^{(e)}] + \sum[\v r_i, \v F_i^{(e)}] = \v M_O &\\
\end{flalign*}
\end{proof}

\begin{cor}
\[ \v M_O = 0 \Rightarrow \v K_O = const \]
\end{cor}
\begin{cor}
\[ M_l = (\v M_O, \v e_l) = 0,~~ \v e_l = const \Rightarrow K_l = const \]
\end{cor}
\begin{proof}
\end{proof}
  \input{2017-10-18}

\begin{cor}
\[ \dot K_l = M_l \]
\end{cor}
\paragraph{Формула преобразования кинетического момента при смене полюса}
% \[ \v K_B = \v K_A + [\v P, \v{AB}] \]
% \begin{proof}
% \begin{flalign*}
% & \v K_B = \sum [\v{BA} + \v \rho_i, m_i\v v_i] = [\v{BA}, m_i\v v_i] + \v K_A = \v K_A + [\v P, \v{AB}] &\\
% \end{flalign*}
% \end{proof}
% \paragraph{Формула преобразовани момента сил при смене полюса}
% \[ \v M_B = \v M_A + [\v F, \v{AB}] \]
% \begin{proof}
% Аналогично.
% \end{proof}

% \begin{teo}
% \[ \dot{\v K_A} = \v M_A + [\v P, \v v_A] \]
% \end{teo}
% \begin{proof}
% \begin{flalign*}
% & \v K_A = \v K_O + [\v P, \v r_A],~~ (\v v_0 \equiv 0) &\\
% & \dot{\v K}_A = \dot{\v K_O} + [\dot {\v P}, \v r_A] + [\v P, \dot {\v r}_A] = \v M_O + [\v F, \v r_A] + [\v P, \v v_A] = &\\
% & \v M_A + [\v P, \v v_A] &\\
% \end{flalign*}
% \end{proof}

% \begin{cor}[Первая теорема Кенига]
% \[ \dot{\v K}_{\text{кен}} = \v M_S \]
% \end{cor}
% \begin{proof}
% \begin{flalign*}
% & \v K_{\text{кен}} = \v K_S;~~ \dot{\v K}_{\text{кен}} = \v{M}_S + [\v P, \v v_S] = \v{M}_S + [m \v v_S, \v v_S] = \v M_S
% \end{flalign*}
% \end{proof}

% \begin{teo}[Об изменении кинетической энергии]
% \[ \dot T = \sum (\v F_u{(e)}, \v v_i) + \sum (\v F_i^{(i)}, \v v_i) \]
% \end{teo}
% \begin{proof}
% \begin{flalign*}
% & T = \frac{1}{2}\sum m_i (\v v_i, \v v_i) &\\
% & \dot T = \sum (\v v_i, m \dot{\v v}_i) = \sum (\v v_i, m \v w_i) = \sum (\v v_i, \v F_i^{(e)} + \v F_i^{(i)}) &\\
% \end{flalign*}
% \end{proof}
% \[ dT = \sum (\v F_i^{(e)}, d\v r_i) + \sum (\v F_i^{(i)}, d\v r_i) \]

% \begin{ass}[Вторая теорема Кенига]
% \[ \v T_{\text{кин}} = \sum (\v F_i, \dot{\v \rho}_i) \]
% \end{ass}
% \begin{proof}
% \begin{flalign*}
% \ldots
% \end{flalign*}
% \end{proof}

% Пусть $\v r_i^{(e)} = - grad_{\v r_i} \Pi(\v r_i, \ldots, \v r_N)$ (внешние силы консервативны).
% \[ \sum(\v F_i^{(e)}, d\v r_i) = - \sum \lb \frac{\partial \Pi}{\partial \v r_i}, d\v r_i \rb = -d\Pi \]
% \[ dT = - d\Pi \Rightarrow d(T + \Pi) = 0 \Rightarrow T + \Pi = const \]

% \begin{teo}[Закон сохранения полной механической энергии]
% Если все внешние силы, действующие на систему консервативны, то полная энергия системы сохраняется.
% \end{teo}
% \section{Основные теоремы динамики в неинерциальных системах отсчета}
% \[ m_i \v w_i = \v F_i^{(e)} + \v F_i^{(i)} + \v F_i^{(\text{пер})} + \v F_i^{(\text{кор})} \]
% \[ \dot{\v P} = \v F + \v F^{\text{пер}} + \v F^{\text{кор}} \]
% \[ \v F^{\text{пер}} = \sum \v F^{\text{пер}} = -\sum m_iw_i^{\text{пер}};~~~ \v F^{\text{кор}} = \sum \v F_i^{\text{кор}} = - \sum m_i\cdot 2\cdot [\v w_\text{кор}, \v v_i] \]
% \[ \dot{\v K}_0 = \v M_O + \v M_O^{\text{пер}} + \v M_O^{\text{кор}} \]
% \[ \v M_O^{\text{кор}} = \sum [\v r_i, \v F_i^{\text{пер}}];~~~ \v M_O^{\text{кор}} = \sum [\v r_i, \v F_i^{\text{кор}}] \]
% \[ \dot T = \sum (F_i, \v v_i) + \sum (\v F_i^{\text{пер}}, \v v_i) + 0 \]
% \[ \sum (\v F_i^{\text{кор}}, \v v_i) = \sum (-2m_i[\v \omega_{\text{пер}}, v_i], \v v_i) = 0 \]

% \begin{xmp}[Система отсчета Кенига]
% \begin{flalign*}
% & \dot{\v K}_S = \dot{\v K}_{\text{кен}} = \v M_S; &\\
% & \dot T_S = \sum(\v F_i, \v v_i);~~~~~ \dot{\v P} = \v F - \sum m_i \v w_s = \v F - m \v w_S &\\
% \end{flalign*}
% \end{xmp}