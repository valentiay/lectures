\documentclass{article}

\usepackage[a4paper, total={6in, 8in}]{geometry}

\usepackage{amsmath}
\usepackage{amsfonts}
\usepackage{amssymb}
\usepackage[T1, T2A]{fontenc}
\usepackage[utf8]{inputenc}
\usepackage[english, russian]{babel}
\usepackage{graphics}
\usepackage{amsthm}
\usepackage{hyperref}
\usepackage{fancyhdr}

\hypersetup{
colorlinks,
citecolor=black,
filecolor=black,
linkcolor=black,
urlcolor=black
}

\geometry{
a4paper,
total={140mm,242mm},
left=35mm,
top=35mm,
}

\renewcommand{\v}[1]{{\overline{#1}}}
\newcommand{\norm}[1]{{\parallel #1 \parallel}}
\newcommand{\abs}[1]{{\vert #1 \vert}}
\newcommand{\ea}{\v{e}_1}
\newcommand{\eb}{\v{e}_2}
\newcommand{\ec}{\v{e}_3}
\newcommand{\ei}{\v{e}_i}
\newcommand{\ek}{\v{e}_k}
\newcommand{\el}{\v{e}_l}
\newcommand{\re}[1]{(\ref{#1})}

\allowdisplaybreaks

\newtheorem*{df}{Определение}
\newtheorem{teo}{Теорема}
\newtheorem{lem}{Лемма}
\newtheorem{prp}{Предложение}
\newtheorem{hyp}{Предположение}
\newtheorem{ass}{Утверждение}
\newtheorem*{cor}{Следствие}
\newtheorem*{ntc}{Замечание}
\newtheorem*{xmp}{Пример}

\renewcommand\qedsymbol{$\blacksquare$}

\newcommand{\lb}{\left(}
\newcommand{\rb}{\right)}


\setcounter{secnumdepth}{0}
\setcounter{tocdepth}{2}

\author{Муницина Мария Александровна}
\title{Аналитическая механика}

\begin{document}
  \label{title}
  
  \begin{titlepage}
  \maketitle
  \begin{center}
  {\itshape\footnotesize Набор: Александр Валентинов}

  {\itshape\footnotesize Об ошибках писать: \url{https://vk.com/valentiay}}
  \end{center}
  \tableofcontents
  \vfill
  \end{titlepage}
  \pagestyle{fancy}
  \fancyhead{}
  \fancyhead[L,R]{\thepage}
  \fancyhead[L]{\hyperref[title]{Аналитическая механика. М.А. Муницина}} 
  \fancyfoot{}

  \input{2017-09-06}
  \input{2017-09-13}
    %2017-09-20
  \subsection{Плоскопараллельное движение}
  \begin{df}
  Движение твердого тела называется плоскопараллельным, если скорости всех точек тела параллельны некоторой неподвижной плоскости:
  $$ \v{v}_{p_i} \parallel \pi,~ \forall p_i \in \text{АТТ} $$
  \end{df}

  $$ \v{v}_{p_i} = \v{v}_{p_j} + [\v{\omega}, \v{p_j p_i}] $$
  $$ 
  (\v{p}_i - \v{v}_{p_i}) = 0 \Leftrightarrow 
  \left[
  \begin{array}{l}
  \v{\omega} = 0 \\
  \v{v}_{p_i} = \v{v}_{p_j},~ \forall p_i, p_j \in \text{АТТ} \\
  \v{\omega} \perp \v{p}_i - \v{v}_{p_i} \parallel \pi \\
  \end{array}
  \right.
  $$
  $$ \vec{v}_{M_i} = \vec{v}_{M_j} + \omega [\vec{\omega}, \overline{M_jM_i}] = \vec{v}_{M_j},~~ \forall M_i, M_j: \overline{M_iM_j} \perp \pi \Rightarrow \vec{w}_{M_i} = \vec{w}_{M_j} $$
  Качение:
  $$ \vec{r}_S = x_S \vec{e}_x + y_S \vec{e}_y $$
  $$ \dot{\vec{e}}_{\xi} = \dot{\varphi}\vec{e}_{\eta},~~ \dot{\vec{e}}_{\eta} = \dot{\varphi}\vec{e}_{\zeta},~~ \dot{\vec{e}}_{\zeta} = 0$$
  $$ \vec{\omega} = \dot{\varphi} \vec{e}_z,~~ \vec{\varepsilon} = \ddot{\varphi} \vec{e}_z \parallel \vec{\omega}$$
  $$ \vec{v}_M = \vec{v}_S + [\vec{\omega}, \overline{SM}] $$
  $$ \vec{w}_M = \vec{w}_S + [\vec{\varepsilon}, \overline{SM}] + [\vec{\omega}, [\vec{\omega}, \overline{SM}]] = \vec{w}_s + [\vec{\varepsilon}, \overline{SM}] - \omega^2 \overline{SM} $$ 

  \begin{teo}
  Если при плоскопараллельном движении угловая скорость твердого тела отлична от нуля, то существует точка, скорость которой равна нулю в данный момент времени.
  \end{teo}
  \begin{proof}
  $$
  \begin{cases}
  \v{v}_c = \v{v}_s + [\v{\omega}, \v{SC}] \\
  \v{v}_c = 0 \\
  \end{cases}
  \Rightarrow
  [\v{\omega}, \v{v}_s] + [\v{\omega}, [\v{\omega}, \v{SC}]] = 0 $$
  $$ [\v{\omega}, \v{v}_s] + \v{\omega}(\v{\omega}, \v{SC}) - \omega^2 \v{SC} = 0 $$
  $$ \v{SC} = \frac{[\v{\omega}, \v{v}_s]}{\omega^2} $$
  \end{proof}
  \begin{cor} 
  Любое плоскопараллельное движение является либо мгновенно-поступательным, либо мгновенно-вращательным
  \end{cor}
  \begin{proof}
  $\v{\omega} = 0$ - мгновенно-поступательное. $\v{\omega}(t) \neq 0$ - вращение вокруг $l$. 
  
  \end{proof}

  \begin{df}
  $C$ - мгновенный центр скоростей
  \end{df}
  \begin{ntc}
  Положение $C$ меняется со временем.
  \end{ntc}
  \begin{xmp}
  Качение без проскальзывания
  \end{xmp}

  \subsection{Тело с неподвижной точкой (вращение вокруг точки)}
  $$ \exists \v{v}_0 \equiv 0 $$
  $$ l \parallel \v{\omega}, O \in l $$
  $$ \v{v}_M = \v{v}_0 + [\v{\omega}, \v{OM}] = 0 + 0,~ \forall M \in l $$
  \begin{df} 
  $l$ - мгновенная ось вращения 
  \end{df}
  $$ \v{v}_p = [\v{\omega}, \v{OP}],~ \v{w_p} = [\v{\varepsilon}, \v{OP}] + \underbrace{[\v{\omega}, [ \v{\omega}, \overline{OP}]]}_{\v{v}_{OC}} $$
  \subsection{Винтовое движение}
  \begin{df} 
  Движение твердого тела называется винтовым, если тело равномерно вращается вокруг неподвижной оси, а скорости всех точек, лежащий на этой оси, равны между собой, постоянны и сонаправленны с осью.
  \end{df}
  \subsection{Общий случай}
  \begin{teo}
  $ \v{\omega} \neq 0 \Rightarrow \exists l:~ \v{\omega} \parallel l,~ \v{v}_{k_i} \parallel l,~ \forall k_i \in l$
  \end{teo}
  \begin{proof}
  $$ \v{\alpha} \perp \v{\omega},~ S \in \alpha $$
  $$
  \begin{cases}
  \v{v}_c = \v{v}_c = \v{v}_s + [\v{\omega}, \v{SC}] \\
  \v{v}_c = \lambda \v{\omega}
  \end{cases}
  \Rightarrow
  0 = [\v{\omega}, \v{v}_s] + [\v{\omega}, [\v{\omega}, \v{SC}]]
  $$
  $$ [ \v{\omega}, \v{v}_s] + \v{\omega}(\v{\omega}, \v{SC}) - \omega^2 \v{SC} = 0 $$
  $$ \v{SC} = \frac{[\v{\omega},\v{v}_c]}{\omega^2} $$
  $$ \exists l: C \in l, l \parallel \v{\omega} $$
  $$ \v{v}_{C_1} = \v{v}_C + [\v{\omega}, \v{CC_1}] = \v{v}_C,~~ \forall C_1 \in l$$ 

  \end{proof}
  $$\v{v}_C = \v{v}_S + \left[ \v{\omega}, \frac{[\v{\omega}, \v{v}_C]}{\omega^2} \right] = \vec{v}_S + \frac{1}{\omega^2}\left(\vec{\omega}(\vec{\omega}, \vec{v}_S) - \omega^2\vec{v}_S \right) = \underbrace{\frac{(\vec{\omega}, \vec{v}_S)}{\omega^2}}_{\lambda} \vec{\omega}$$
  $$\lambda = \frac{(\vec{\omega}, \vec{v}_S)}{\omega^2} \text{ - параметр (шаг винта).}$$

  \begin{cor}
  Любое движение твердого тела является в каждый момент времени либо мгновенно-поступательным ($\omega = 0$, $\lambda \rightarrow +\infty$), либо мгновенно-вращательным ($\omega \neq 0$, $\lambda = 0$) , либо мгновенно-винтовым ($\omega \neq 0$, $\lambda \neq 0$).
  \end{cor}
  \begin{df}
  $\{l, \v{\omega}, \v{v}\}$ - кинематический винт.
  \end{df}
  $$\v{v}_S = v_x\v{e}_x + v_y\v{e}_y + v_z\v{e}_z$$
  $$\v{r}_S = x_S\v{e}_x + y_S\v{e}_y + z_S\v{e}_z$$
  $$\v{\omega} = \omega_x\v{e}_x + \omega_y\v{e}_y + \omega_z\v{e}_z$$
  $$ \v{r}_C = x\v{e}_x + y\v{e}_y + z\v{e}_z $$
  $$ \v{v}_S + [\v{\omega}, \v{SC}] = \lambda \v{\omega} \Rightarrow \lambda = \frac{v_x + \omega_y(z - z_S) - \omega_z(y - y_S)}{\omega_x} = $$
  $$ = \frac{v_y + \omega_z(x - x_S) - \omega_x(z - z_S)}{\omega_y} = \frac{v_z + \omega_x(y - y_S) - \omega_y(x - x_S)}{\omega_z} $$
  \section{Кинематика сложного движения}
  $OXYZ$ - неподвижная система отсчета ($\v{r})$, $O_1\xi\eta\zeta$ - подвижная система отсчета ($\v{\rho}$).

  $$ \v{u} = u_x \v{e}_x + u_y \v{e}_y + u_z \v{e}_z $$
  $$ \v{u} = u_{\xi} \v{e}_{\xi} + u_{\eta} \v{e}_{\eta} + u_{\zeta} \v{e}_{\zeta} $$
  $$ \frac{d\v{u}}{dt} = \dot {u_x} \v{e}_x + \dot{u_y} \v{e}_y + \dot{u_z} \v{e}_z \text{ - абсолютная производная} $$
  $$ \dot{\v u} = \dot{u_{\xi}} \v{e}_{\xi} + \dot{u_{\eta}} \v{e}_{\eta} + \dot{u_{\zeta}} \v{e}_{\zeta} \text{ - относительная производная}$$
  \begin{teo}(Связь абсолютной и относительной производной) 
  $\frac{d\v{u}}{dt} = \dot{\v{u}} + [\v{\omega}, \v{u}]$, где $\v{\omega}$ - угловая скорость $O_1\xi\eta\zeta$ относительно $OXYZ$
  \end{teo}
  \begin{proof}
  $$ \frac{du}{dt} = \dot{u}_{\xi}\vec{e}_{\xi} + \dot{u}_{\eta}\vec{e}_{\eta} + \dot{u}_{\zeta}\vec{e}_{\zeta} + u_{\xi}\frac{d\vec{e}_{\xi}}{dt} + u_{\eta}\frac{d\vec{e}_{\eta}}{dt} + u_{\zeta}\frac{d\vec{e}_{\zeta}}{dt} = $$
  $$ = \dot{\vec{u}} + u_{\xi}[\vec{\omega}, \vec{e}_{\xi}] + u_{\eta}[\vec{\omega}, \vec{e}_{\eta}] + u_{\zeta}[\vec{\omega}, \vec{e}_{\zeta}] = \dot{\vec{u}} + [\vec{\omega}, \vec{u}] $$
  $$ \left(\frac{d\vec{e}_i}{dt} = [\vec{\omega}, \vec{e}_i] - \text{ - формула Пуассона},~~ \dot{\vec{e}}_i = 0\right) $$
  \end{proof}
  \subsection{Сложное движение материальной точки}
  \begin{df}
  Абсолютной скоростью материальной точки называется ее скорость относительно неподвижной системы отсчета. $\v{v}_{\text{абс}} = \frac{d}{dt}\v{r}$
  \end{df}
  \begin{df}
  Относительной скоростью материальной точки называется ее скорость относительно подвижной системы отсчета. $\v{v}_{\text{отн}} = \dot{\v{\rho}}$
  \end{df}
  \begin{df}
  Переносной скоростью материальной точки называется абсолютная скорость той точки подвижной системы отсчета, в которой находится движующаяся точка в данный момент времени.
  \end{df}
  \begin{teo}
  [Формула сложения скоростей] $\v{v}_{\text{абс}} = \v{v}_{\text{отн}} + \v{v}_{\text{пер}}$
  \end{teo}
  \begin{proof}
  $$ \v{v}_{\text{абс}} = \frac{d}{dt}(\v{R} + \v{\rho}) = \frac{dR}{dt} + \dot{\v{\rho}} + [\v{\omega}, \v{\rho}] = $$
  $$ = \v{v}_{O_1} + \v{v}_{\text{отн}} + [\v{\omega}, \v{\rho}] = \v{v}_{\text{отн}} + \v{v}_{\text{пер}} $$
  \end{proof}
  \begin{df}
  Абсолютным ускорением материальной точки называется ее ускорение относительно неподвижной системы отсчета. $\v{w}_{\text{абс}} = \frac{d}{dt}\v{v}_{\text{абс}}$
  \end{df}
  \begin{df}
  Относительным ускорением материальной точки называется ее ускорение относительно подвижной системы отсчета. $\v{w}_{\text{отн}} = \dot{\v{v}_{\text{отн}}}$
  \end{df}
  \begin{df}
  $ \vec{\omega}_{\text{пер}} = \vec{\omega}_{O_1} + [\vec{\varepsilon}, \vec{\rho}] + [\vec{\omega}, [\vec{\omega}, \vec{\rho}]] $
  \end{df}
  \begin{df}
  $ \vec{\omega}_{\text{кор}} = 2[\vec{\omega}, \vec{v}_\text{отн}] $
  \end{df}
  \begin{teo}
  [Формула сложения ускорений] $\v{w}_{\text{абс}} = \v{w}_{\text{отн}} + \v{w}_{\text{пер}} + \v{w}_{\text{кор}}$
  \end{teo}

  \begin{proof}
  $$ \vec{w}_{\text{абс}} = \frac{d}{dt}(\vec{v}_{\text{отн}} + \vec{v}_{\text{пер}}) = \frac{d}{dt} (\vec{v}_{\text{отн}} + \vec{v}_{O_1} + [\vec{\omega}, \vec{\rho}]) = $$ 
  $$ = \dot{\vec{v}}_{\text{отн}} + [\vec{\omega}, \vec{v}_{\text{отн}}] + \frac{d}{dt}\vec{v}_{O_1} + \left[\frac{d\vec{\omega}}{dt}, \vec{\rho}\right] + [\vec{\omega}, \vec{\rho} + [\vec{\omega}, \vec{\rho}]] = $$ 
  $$ = \dot{\vec{v}}_{\text{отн}} + \dot{\vec{v}}_{O_1} + [\vec{\varepsilon}, \vec{\rho}] + 2[\vec{\omega}, \vec{v}_{\text{отн}}] + [\vec{\omega}, [\vec{\omega}, \vec{\rho}]] $$
  \end{proof}
    %2017-09-27
  \subsection{Сложное движение твердого тела}
  Рассмотрим неподвижную систему отсчета $OXYZ$, подвижную $O_1xyz$, и систему, связанную с телом $S\xi\eta\zeta$.
  \begin{df} Абсолютная угловая скорость - угловая скорость $S\xi\eta\zeta$ относительно $OXYZ$\end{df}
  \begin{df} Относительная угловая скорость - угловая скорость $S\xi\eta\zeta$ относительно $O_1xyz$\end{df}
  \begin{df} Переносная угловая скорость - угловая скорость $Oxyz$ относительно $OXYZ$\end{df}  
  \begin{teo}[О сложении угловых скоростей] $\vec{\omega}_{\text{абс}} = \vec{\omega}_{\text{отн}} + \vec{\omega}_{\text{пер}}$ \end{teo}
  \begin{proof}
  $$ \vec{v}_A^{\text{абс}} = \vec{v}_A^{\text{отн}} + \vec{v}_A^{\text{пер}} $$
  $$ \vec{v}_B^{\text{абс}} = \vec{v}_B^{\text{отн}} + \vec{v}_B^{\text{пер}} $$
  $$ \vec{v}_B^{\text{абс}} = \vec{v}_A^\text{абс} + [\vec{\omega}_{\text{абс}}, \overline{AB}] $$

  $$ \vec{v}_B^{\text{отн}} = \vec{v}_A^\text{отн} + [\vec{\omega}_{\text{отн}}, \overline{AB}] $$

  $$ \vec{v}_B^{\text{пер}} = \vec{v}_A^\text{пер} + [\vec{\omega}_{\text{пер}}, \overline{AB}] $$
  $$ \Rightarrow 0 = 0 + [\vec{\omega}_{\text{абс}} - \vec{\omega}_{\text{отн}} - \vec{\omega}_{\text{пер}}, \overline{AB}] = 0,~~ \forall \overline{AB} \Leftrightarrow \vec{\omega}_{\text{абс}} = \vec{\omega}_{\text{отн}} + \vec{\omega}_{\text{пер}} $$
  \end{proof}

  \begin{ntc}
  $\frac{d\vec{\omega}_{\text{пер}}}{dt} = \dot{\vec{\omega}}_\text{пер} + [\vec{\omega}_\text{пер}, \vec{\omega}_\text{пер}] = \dot{\vec{\omega}}_\text{пер} $
  \end{ntc}

  \begin{teo}[О сложении угловых ускорений] 
  $\vec{\varepsilon}_{\text{абс}} = \vec{\varepsilon}_{\text{отн}} + \vec{\varepsilon}_{\text{пер}} + [\vec{\omega}_{\text{пер}}, \vec{\omega}_\text{отн}]$, где $\vec{\varepsilon}_{\text{абс}} = \frac{d}{dt}\vec{\omega}_{\text{абс}}$, $\vec{\varepsilon}_{\text{отн}} = \dot{\vec{\omega}}_{\text{отн}}$, $\vec{\varepsilon}_{\text{пер}} = \frac{d}{dt}\vec{\omega}_{\text{пер}} = \dot{\vec{\omega}}_{\text{пер}}$
  \end{teo}

  \begin{proof}
  $$ \vec{\varepsilon}_{\text{абс}} = \frac{d}{dt}(\vec{\omega}_{\text{отн}} + \vec{\omega}_{\text{пер}}) = $$ 
  $$ = \dot{\vec{\omega}}_{\text{отн}} + [\vec{\omega}_{\text{пер}}, \vec{\omega}_{\text{отн}}] + \frac{d}{dt}\vec{\omega}_{\text{пер}} =
  \vec{\varepsilon}_{\text{отн}} + [\vec{\omega}_{\text{пер}}, \vec{\omega}_{\text{отн}}] + \vec{\varepsilon}_{\text{пер}} $$ 

  \end{proof}

  \subparagraph{Несколько подвижных сисем отсчета}~
  
  $OXYZ$ - неподвижная СО
  
  $Ox_1y_1z_1$, $Ox_2y_2z_2$ , $\ldots Ox_ny_nz_n$ - подвижные СО
  
  $S\xi\eta\zeta$ - связана с телом
  
  $\vec{\omega}$ - угловая скорость $S\xi\eta\zeta$ относительно $OXYZ$

  Тогда: $\vec{\omega} = \sum\limits_{i = 1}^{n} \vec{\omega_i}$
  
  \subsection{Кинематические формулы Эйлера} 
  \begin{df} $ Ox = (OXY)\cap(O\xi\eta) $ - линия узлов \end{df}
  \begin{df} $\psi = \angle(Ox, OX)$ - угол прецессии \end{df}
  \begin{df} $\Theta = \angle (O\zeta, OZ)$ - угол нутации \end{df}
  \begin{df} $\varphi = \angle (Ox, O\xi)$ - угол нутации \end{df}
  \begin{df} $\{\psi, \Theta, \varphi\}$ - углы Эйлера \end{df}

  Повороты:
  $ OXYZ \xrightarrow{\psi, OZ} OxyZ \xrightarrow{\Theta, Ox} Oxy\zeta \xrightarrow{\varphi, O\zeta} O\xi\eta\zeta $

  $\vec{\omega} = \dot{\psi}\vec{e}_z + \dot{\Theta}\vec{e}_x + \dot{\varphi}\vec{e}_{\zeta}$

  $\vec{e}_x = \cos \varphi \vec{e}_{\xi} + \sin \varphi \vec{e}_{\eta}$
  
  $\vec{e}_z = \cos \Theta \vec{e}_{\zeta} + \sin \Theta ( \sin \varphi \vec{e}_{\xi} + \cos \varphi \vec{e}_{\eta} )$
  $$\begin{array}{rcl}\vec{\omega} & = & \dot\psi(\sin \Theta \sin \varphi \vec{e}_{\xi} + \sin \Theta \cos \varphi \vec{e}_{\eta} + \cos \Theta \vec{e}_{\zeta}) \\
  & + & \dot \Theta (\cos \varphi \vec{e}_{\xi} - \sin \vec{e}_{\eta}) \\
  & + & \dot{\varphi}\vec{e}_{\zeta} = \omega_{\xi}\vec{e}_{\xi} + \omega_{\eta}\vec{e}_{\eta} + \omega_{\zeta}\vec{e}_{\zeta} \\
  \end{array}$$

  $$
  \begin{cases}
  \vec{\omega}_{\xi} = \dot{\psi}\sin\Theta \sin\varphi + \dot{\Theta}\cos\varphi \\
  \vec{\omega}_{\eta} = \dot{\psi}\sin\Theta \cos\varphi + \dot{\Theta}\sin\varphi \\
  \vec{\omega}_{\zeta} = \dot{\psi}\cos\Theta + \dot{\varphi} \\
  \end{cases}
  \text{ - кинематические формулы Эйлера}
  $$

  \begin{df} 
  Движение твердого тела называется прецессией, если некоторая ось, неподвижная в теле, в абсолютном пространстве движется по поверхности неподвижного кругового конуса. $\dot{\Theta} = 0$. Если $\dot {\psi} = const$, $\dot {\varphi} = const$, то прецессия называется регулярной.
  \end{df}

  \section{Алгебра кватернионов}
  \begin{df} Алгеброй над полем называется векторное пространство над этим полем, снабженное билинейной операцией умножения. \end{df}
  \begin{xmp} ~

  $\underline{n=2}$(\textit{Комплексные числа}). $z_1 = a + bi$, $z_2 = c + di$ 

  $$ z_1z_2 = (ac - bd) + (ad + bc)i $$

  \end{xmp}
  $\underline{n=4}$(\textit{Алгебра кватернионов})

  \begin{flalign*}
  & \Lambda = \lambda_0 \vec{i}_0 + \lambda_1 \vec{i}_1 + \lambda_2 \vec{i}_2 + \lambda_3 \vec{i}_3 \in \mathbb{H} &\\
  & \{\vec{i}_0, \vec{i}_1, \vec{i}_2, \vec{i}_3\} \text{ - базис} &\\ 
  & \Lambda = \lambda_0 + \overline{\lambda} &\\
  & i_0 \circ i_k = i_k k = \overline{1, 3},~ i_0 \circ i_0 = 1 &\\
  \end{flalign*}
  \begin{flalign*}
  & i_k \circ i_m = -(i_k, i_m) + [i_k, i_m] k, m \in \{1,2,3\} &\\
  & \overline{\lambda} \circ \overline{\mu} = (\lambda_1 \vec{i}_1 + \lambda_2 \vec{i}_2 + \lambda_3 \vec{i}_3) \circ (\mu_1 \vec{i}_1 + \mu_2 \vec{i}_2 + \mu_3 \vec{i}_3) = -(\overline{\lambda}, \overline{\mu}) + [\overline{\lambda}, \overline{\mu}] &\\
  & \Lambda \circ M = (\lambda + \overline{\lambda}) \circ (\mu + \overline{\mu}) = \lambda_0 \mu_0 + \lambda_0\overline{\mu} + \overline{\lambda}\mu_0 - (\overline{\lambda}, \overline{\mu}) + [\overline{\lambda}, \overline{\mu}] &\\
  \end{flalign*}
  \paragraph{Свойства:}
  \begin{enumerate}
    \item $(\Lambda \circ M) \circ N = \Lambda \circ (M \circ N)$
    \item $(\Lambda + M) \circ N = \Lambda \circ N + M \circ N $
    \item $\Lambda \circ M \neq M \circ \Lambda$
  \end{enumerate}
  %2017-10-04
  \begin{df}
  \[ \overline{\Lambda} = \lambda_0 - \overline{\lambda} \]
  \end{df}
  \begin{ass}
  \[ \overline{\Lambda \circ M} = \overline{M} \circ \overline{\Lambda} \]
  \end{ass}
  \begin{proof}
  \[ \overline{\Lambda \circ M} = \lambda_0\mu_0 - (\overline\lambda, \overline\mu) - \lambda_0\overline\mu - \mu_0\overline\lambda - [\overline\lambda, \overline\mu] = \]
  \[ = (\mu_0 - \overline\mu) \circ (\lambda_0 - \overline\lambda) = \overline M \circ \overline \Lambda \]
  \end{proof}
  \begin{df}
  \[ \parallel \Lambda \parallel = \Lambda \circ \overline \Lambda = (\lambda_0 + \overline \lambda) \circ (\lambda_0 - \overline \lambda) = \lambda_0^2 + \overline \lambda^2 = \sum\limits_{k = 0}^3 \lambda_k^2 = | \Lambda |^2 \text{ - норма $\Lambda$ }\]
  \end{df}
  \begin{ass}
  \[ \parallel \Lambda \circ M \parallel = \norm{\Lambda} \cdot \norm{M} \]
  \end{ass}
  \begin{proof}
  \[ \norm{\Lambda \circ M} = (\Lambda \circ M) \circ (\overline{\Lambda \circ M}) = \Lambda \circ \underbrace{M \circ \overline M}_{\norm{M}}  \circ \overline \Lambda = \norm{M} \cdot \norm{\Lambda} \]
  \end{proof}
  \begin{df}
  \[ \Lambda^{-1} = \frac{\overline \Lambda}{\norm{\Lambda}},~ \norm \Lambda \neq 0 \]
  \end{df}
  \begin{ntc}
  \[ \Lambda \circ \frac{\overline{\Lambda}}{\norm{\Lambda}} = \frac{\overline \Lambda}{\norm \Lambda} \circ \Lambda = \frac{\norm \Lambda}{\norm \Lambda} = 1 \]
  \end{ntc}

  \paragraph*{Формула Муавра}
  \begin{flalign*}
  & \Lambda = \lambda_0 + \overline \lambda =  \abs \Lambda \lb \frac{\lambda_0}{\abs \Lambda} + \frac{\overline \lambda}{\abs \lambda } \frac{\abs {\overline \lambda}}{\abs \Lambda} \rb = \abs \Lambda \lb \cos \nu + \overline e \sin \nu \rb &\\
  & \overline e = \frac{\overline \lambda}{\abs {\overline \lambda}},~ \cos\nu = \frac{\lambda_0}{\Lambda},~ \sin\nu = \frac{\overline \lambda}{\abs{\Lambda}} &\\
  & \Lambda_1 = \abs{\Lambda_1}(\cos \nu_1 + \overline e \sin \nu_1) &\\
  & \Lambda_2 = \abs{\Lambda_2}(\cos \nu_2 + \overline e \sin \nu_2) &\\
  & \Lambda_1 \circ \Lambda_2 = \abs{\Lambda_1}\cdot\abs{\Lambda_2}(\cos\nu_1 \cos\nu_2 - \sin\nu_1\sin\nu_2(\overline e, \overline e) + \cos\nu_1\sin\nu_2 \overline e + &\\ 
  & + \cos\nu_2\sin\nu_1\overline e + \sin\nu_2\sin\nu_2[\overline e, \overline e]) = \abs{\Lambda_1}\abs{\Lambda_2} \cdot (\cos(\nu_1 + \nu_2) + \overline e \sin(\nu_1 + \nu_2)) &\\
  \end{flalign*}
  \[ \Lambda^k = \abs \Lambda^k \cdot (\cos k\nu + \overline e \sin k\nu) \text{ --- формула Муавра } \]
  \section{Задание ориентации твердого тела с помощью кватернионов}
  $E = \{\ea, \eb, \ec\}$ --- неподвижный базис \\
  $E' = \{\ea', \eb', \ec' \}$ --- связанный с телом
  \begin{teo}
  Произвольному положению твердого тела с неподвижной точкой соответсвует номированный кватернион, удовлетворяющий равенству:
  \[ \v{e}_i = \Lambda \circ \v{e}_i \circ \overline \Lambda,~~ i = 1\ldots3  \]
  \end{teo}
  \begin{ntc}
  $\Lambda$ --- нормирован, если $\norm \Lambda = 1$
  \end{ntc}
  \begin{proof}~
  \begin{enumerate}
  \item Нормированность
  \[ \norm{\ei'} = \norm \Lambda \cdot \norm{\ei} \cdot \norm{\overline{\Lambda}} \Rightarrow 1 = \norm \Lambda \cdot 1 \cdot \norm \Lambda \Rightarrow \norm \Lambda = 1 \]
  \item Существование решения. $\Lambda = \lambda_0 + \overline \lambda$
  \begin{flalign*}
  & 
  \begin{cases} 
  \lambda_0^2 + \overline \lambda^2 = 1  \\
  \ei' \circ \Lambda = \Lambda \circ \ei \\
  \end{cases}
  &
  \begin{cases}
  \lambda_0^2 + \overline \lambda^2 = 1 \\
  \ei' \circ (\lambda_0 + \overline \lambda) = (\lambda_0 + \overline \lambda) \circ \ei \\
  \end{cases}
  \end{flalign*}
  \begin{flalign*}
  &
  \begin{cases}
  \lambda_0 \ei' - (\ei', \overline \lambda) + [\ei', \overline \lambda] = \lambda_0\ei' - (\lambda, \ei') + [\overline \lambda, \ei] \\
  \lambda_0^2 + \overline \lambda^2 = 1 \\
  \end{cases}
  &\\
  \end{flalign*}
  \begin{flalign*}
  & \begin{cases}
  \lambda_0^2 + \overline \lambda^2= 1  \\
  (\overline \lambda, \overline r_i) = 0 \\
  \lambda_0 \overline r_i - [\overline \lambda, \overline s_i] = 0 \\
  \end{cases}
  &
  \overline r_i = \ei' - \ei,~ \overline s_i = \ei' + \ei & ~~i = 1 \ldots 3
  &\\
  \end{flalign*}
  \begin{enumerate}
  \item 
  \begin{flalign*}
  & (\overline r_k, \overline s_k) = (\ek' - \ek, \ek' + \ek) = (\ek', \ek') - (\ek, \ek) = 0 &\\
  & (\overline r_k, \overline s_l) = (\ek' - \ek, \el' + \el) = (\ek', \el') + (\ek', \el) -  (\ek, \el') - (\ek, \el) = &\\
  & = -(\el' - \el, \ek' + \ek) = -(\overline s_k, \overline r_l),~ k \neq 1 &\\
  \end{flalign*}
  \item \label{two.o}
  \begin{flalign*}
  & (\overline r_1, \overline r_2, \overline r_3) = (\ea' - \ea, \eb' - \eb, \ec' - \ec) = (\ea', \eb', \ec') - (\ea, \eb, \ec) - &\\ 
  & - (\ea', \eb', \ec) + (\ea, \eb, \ec') =  1 - 1 - (\underbrace{[\ea', \eb']}_{\ec'}, \ec) + (\underbrace{[\ea, \eb]}_{\ec}, \ec') = 0 &\\
  \end{flalign*}
  \item 
  \begin{flalign*}
  & \overline r_1(\overline s_2, \overline r_3) + \overline r_2(\overline s_3, \overline r_1) + \overline r_3(\overline s_1, \overline r_2) &\\
  & \re{two.o} \Rightarrow c_1\v r_1 + c_2\v r_2 + c_3\v r_3 = 0 &\\
  & 
  \begin{cases}
  0 + c_2(\v s_1, \v r_2) - c_3(\v s_2, \v r_1) = 0 \\
  -c_1(\v s_1, \v r_2) + 0 + c_3(\v s_2, \v r_3) = 0 \\
  c_1(\v s_3, \v r_1) - c_2 (\v s_2, \v r_3) + 0 = 0 \\    
  \end{cases}
  &\\
  &
  \begin{cases}
  c_1 = (\v s_2, \v r_3) \\
  c_2 = (\v s_3, \v r_1) \\
  c_3 = (\v s_1, \v r_2) \\
  \end{cases}
  \begin{cases}
  \lambda_0^2 + \lambda^2 = 1 \\
  (\v r_k, \v \lambda) = 0 \\
  \lambda_0\v r_k + [\v s_k, \v \lambda] = 0 \\
  \end{cases}
  \begin{array}{c}
  (1) \\ (2) \\ (3) \\
  \end{array}
  &\\
  &
  (3) \Leftrightarrow
  \begin{cases}
  \lambda_0\v r_1 + [\v s_1, \alpha[\v r_1, \v r_2]] = 0 \\
  \lambda_0\v r_2 + [\v s_2, \alpha[\v r_1, \v r_2]] = 0 \\
  \lambda_0\v r_3 + [\v s_3, \alpha[\v r_1, \v r_2]] = 0 \\
  \end{cases}
  &\\
  &
  \begin{cases}
  \lambda_0 \v r_1 + \alpha \v r_1 (\v s_1, \v r_1) - 0 = 0 \\
  \lambda_0 \v r_2 + 0 - \alpha\v r_2(\v s_2, \v r_1) = 0 \\
  \lambda_0 \v r_3 + \alpha r_1(\v s_3, \v r_2) - \alpha r_2(\v s_3, \v r_1) = 0 \\
  \end{cases}
  &\\
  &
  \begin{cases}
  \lambda_0\v r_1 + \alpha\v r_1(\v s_1,\v r_2) = 0 \\
  \lambda_0\v r_2 + \alpha\v r_2(\v s_1,\v r_2) = 0 \\
  \lambda_0\v r_3 + \alpha\v r_3(\v s_1,\v r_2) = 0 \\
  \end{cases}
  &
  \lambda_0 = -\alpha(\v s_1, \v r_2) = \alpha(\v s_2, \v r_1)
  &\\
  &
  (1) \Rightarrow \alpha^2((\v s_2, \v r_1)^2 + [\v r_1, \v e_2]^2)^2 = 1 \Rightarrow & \alpha = \pm \frac{1}{\sqrt{(\v s_2, \v r_1)^2 + [\v r_1, \v r_2]^2}} 
  &\\
  \end{flalign*}
  \[ \Lambda = \pm \frac{(\v s_2, \v r_1) + [\v r_1, \v r_2]}{\sqrt{(\v s_2, \v r_1)^2 + [\v r_1, \v r_2]^2}} \]
  \end{enumerate}
  \end{enumerate}
  \end{proof}
  \begin{df}
  \[ f(M) = \Lambda \circ M \circ \overline \Lambda;~~ M \rightarrow f(M),~~ \norm{\Lambda} = 1 \text{--- присоединенное преобразование} \]
  \end{df}
  \begin{ass}
  Присоединенное преобразование не меняет скалярные части кватернионов и модуль векторной части
  \end{ass}
  \begin{proof}~
  \begin{enumerate}
  \item
  $ f(M) = \Lambda \circ (\mu_0 + \overline \mu) \circ \overline \Lambda = \Lambda \circ \mu_0 \circ \overline \Lambda + \Lambda \circ \overline \mu \circ \Lambda = \mu_0 \norm \Lambda + f(\v \mu) = \mu_0 + \v \mu ' $
  \item $\mu_0^2 + \v \mu^2 = \norm M = \norm{\Lambda \circ M \circ \v \Lambda} = \norm{f(M)} = \mu_0^2 + \v \mu'^2 \Rightarrow \mu^2 = \v \mu'^2$
  \end{enumerate}
  \end{proof}
  \begin{cor}
  Всегда существует присоединенное преобразование, переводящее орты неподвижного базиса в орты базиса, связанного с телом.
  \end{cor}
  \begin{proof}
  \begin{flalign}
  & \ei' = \Lambda \circ \ei \circ \v \Lambda = f(\ei) &\\
  & \v r = \sum\limits_{k = 1}^3 r_k \ek,~~ f(r) = \Lambda \circ \sum r_k \ek \v \Lambda = \sum\limits_k r_k f(\ek) = \sum\limits_k = r_k\ek = \v r ' &\\
  \end{flalign}
  \end{proof}
  \begin{equation}
  \label{b.star}
  \boxed{\v r' = \Lambda \circ \v r \circ \v \Lambda}
  \end{equation}
  \begin{cor}
  При повороте твердого тела вокруг неподвижной точки справедлива \re{b.star}, где $\v r$ -- начальное положение точки, $\v r'$ -- ее положение после поворота, а $\Lambda$ -- кватернион соответствующего преобразования.
  \end{cor}
  \input{2017-10-11}
  \input{2017-10-18}
  \input{2017-10-25}
  \section{Движение в центральном поле}
\subsection{Законы сохранения}
В центральном поле
\[
	m \ddot{ \v r} = \v F,\quad \v F = F(r) \frac{\v r}{r}
\]

\paragraph*{Закон сохранения энергии:}
\[
	\Pi = -\int F(r) dr,~~~ \frac{\partial \Pi}{\partial t} = 0 \Rightarrow T + \Pi = h = const
\]
\paragraph*{Закон сохранения кинетического момента:}
\[
	\v M_O = \left[ \v r, F(r) \frac{\v r}{r} \right] = 0 \Rightarrow \dot{\v k}_O = 0 \Rightarrow \v k_O = [\v r, \v m\v v] = \v k = const
\]
\begin{cor}
Траектория точки в центральном поле всегда является плоской кривой.
\end{cor}
\begin{proof}
\[
	[\v r, m\v v] = \v k \perp \alpha \Rightarrow \v r \in \alpha \quad \forall t,\,\alpha = const
\]
\end{proof}

\begin{cor}
\[
	r^2 \dot \varphi = c = const
\]
\end{cor}
\begin{proof}
\[
	| \v k | = |[ \v r, m \v v]| = |[r \v e_r,\, m (\dot {r } \v e_r + r \dot \varphi \v e_\varphi)] | = mr^2|\dot \varphi||\v e_z| = const \Rightarrow r^2\dot \varphi = const
\]
\end{proof}

\paragraph*{Геометрический смысл}
\begin{flalign*}
& S = \iint dS = \int\limits_{\varphi_0}^\varphi d\varphi\int\limits_0^{r(\varphi)} rdr = \int\limits_{\varphi_0}^\varphi \frac{r^2(\varphi)}{2}d \varphi &\\
& \dot S = \frac{dS}{d\varphi}, \quad \dot \varphi = \frac{r^2}{2}\varphi = \frac{c}{2} = const &\\
& \sigma = \dot s = \frac{c}{2} \text{ --- секториальная скорость}
\end{flalign*}

\subsection{Формулы Бине}

\begin{teo}[Формулы Бине]
При движении точи в центральном поле справедливы следующие равенства:
\[
	v^2 = c^2 \left( \left[ \frac{d}{d\varphi}\left( \frac{1}{r} \right) \right]^2 + \frac{1}{r^2} \right)
\]
\[
	F = - \frac{mc^2}{r^2} \left( \frac{d^2}{d\varphi^2}\left( \frac{1}{r} \right) + \frac{1}{r}\right)
\]
\end{teo}
\begin{proof}
\begin{flalign*}
& v^2 = \dot r^2 + r^2 \dot \varphi^2 &\\
& \v w = (\ddot r - r \dot \varphi^2)\v e^r + (r \ddot \varphi + 2\dot r \dot \varphi)\v e_\varphi &\\
& m \v w = F \v e_r \quad 
\begin{cases}
m(\ddot r - r \dot \varphi) = F \\
r \ddot \varphi + 2 \dot r \dot \varphi = 0 \Rightarrow \frac{d}{dt}(r^2 \dot \varphi) = 0 \\
\end{cases} &\\
& \dot r = \frac{dr}{d\varphi} \quad \dot \varphi = \frac{dr}{d\varphi}\frac{c}{r^2} = -c \frac{d}{d\varphi}\left( \frac{1}{r} \right) &\\
& \ddot r = \frac{d \dot r}{d \varphi}\dot \varphi = -\frac{c^2}{r^2} \frac{d^2}{d\varphi^2}\left( \frac{1}{r} \right) &\\
& v^2 = c^2 \left[ \frac{d}{d\varphi}\left( \frac{1}{r} \right) \right]^2 + r^2 \frac{c^2}{r^4} = c^2 \left( \left[ \frac{d}{d\varphi} \left( \frac{1}{r} \right)\right]^2 + \frac{1}{r^2} \right) &\\
& F = - \frac{mc^2}{r^2} \left( \frac{d^2}{d\varphi^2}\left( \frac{1}{r} \right) + \frac{1}{r}\right) &\\
\end{flalign*}
\end{proof}

\noindent Определим траекторию.
\begin{flalign*}
& T + \Pi = h, \quad T = \frac{m}{2}v^2 &\\
& \frac{mc^2}{2} \left[ \frac{d}{d\varphi}\left( \frac{1}{r} \right) \right]^2 + \underbrace{\frac{mc^2}{2r^2} + \Pi(r)}_{\Pi_c(r)} = h &\\
& \pm \sqrt{\frac{mc^2}{2}}\frac{d}{d\varphi}\left( \frac{1}{r} \right) = \sqrt{h - \Pi_c(r)} &\\
& \text{Замена: } \frac{1}{r} = u \quad \pm \sqrt{\frac{mc^2}{2}} \int\limits_{1/r_0}^{1/r} \frac{du}{\sqrt{h - \Pi_c(u)}} = \varphi - \varphi_0 \Rightarrow r(\varphi) &\\
& \dot \varphi = \frac{c}{r^2(\varphi)} \Rightarrow \int\limits_{\varphi_0}^\varphi r^2(\varphi) d \varphi = \int\limits_{t_0}^t cdt = c(t - t_0)
\end{flalign*}

\subsection{Движение точки в центральном гравитационном поле}
\[
	F = -\gamma \frac{mM}{r^2}, \quad \Pi(r) = -\gamma \frac{mM}{r}
\]
\begin{flalign*}
\varphi & = \pm \sqrt{\frac{mc^2}{2}}\int \frac{du}{\sqrt{h - m\frac{c^2}{2u^2} + \gamma mM u}} = \pm \int\frac{du}{\sqrt{\frac{2h}{mc^2} - u^2 + \frac{2\gamma M}{c^2}u}} = &\\
& = \pm \int \frac{du}{\sqrt{\frac{2h}{mc^2} + \frac{\gamma^2M^2}{c^4} - \left( u - \frac{\gamma M}{c^2} \right)^2}} = \pm \arccos \frac{\frac{1}{r} - \frac{\gamma M}{c^2}}{\sqrt{\frac{2h}{mc^2} + \frac{\gamma^2M^2}{c^4}}} + \varphi_0 &\\
& \frac{1}{r} = \frac{\gamma M}{c^2} + \sqrt{\frac{2h}{mc^2} + \frac{\gamma^2 M}{c^4}}\cos(\varphi - \varphi_0) &\\
& \frac{c^2}{\gamma m} = p, \quad \sqrt{\frac{2h}{mc^2}p^2 + 1} = e \Rightarrow r = \frac{p}{1 + e\cos(\varphi - \varphi_0)}
\end{flalign*}
То есть $\varphi_0$ зависит от $c$ и $h$.

\begin{ntc}
$\varphi_0 = 0 \quad (\varphi' = \varphi - \varphi_0)$
\end{ntc}

\begin{ass}
Траектория точки в центральном гравитационном поле является коническим сечением.
\begin{itemize}
\item $e = 0$: $\left( h^* := h = - \frac{mc^2}{2p^2} = - \frac{m\gamma^2M^2}{2c^2} \right)$ --- окружность.
\item $0 < e < 1$: $\left( h^* < h < 0 \right)$ --- эллипс.
\item $e = 1$: $\left( h = 0 \right)$ --- парабола.
\item $e > 1$: $\left( h > 0 \right)$ --- гипербола.
\end{itemize}
\end{ass}

\begin{xmp}[Первая космическая скорость]
\begin{flalign*}
& v_1 =\; ? &\\
& \frac{mv^2}{2} - \gamma \frac{mM}{R} = -\frac{m\gamma^2M^2}{2c^2} = - \frac{m\gamma^2 M^2}{2R^2v_1^2} &\\
& c = R^2 \dot \varphi = R v_1 \text{ (окружность)} &\\
& v_1^2 - \frac{2\gamma M}{R} + \frac{\gamma^2 M^2}{R^2v_1^2} = 0 &\\
& \left( v_1 - \frac{\gamma M}{R v_1} \right)^2 = 0 \Rightarrow v_1 = \sqrt{\frac{\gamma M}{R}} &\\
\end{flalign*}
\end{xmp}

\begin{xmp}[Вторая космическая скорость]
\begin{flalign*}
& \frac{mv_2^2}{2} - \frac{\gamma mM}{R} = 0 \Rightarrow v_2^2 = \frac{2\gamma M}{R} &\\
\end{flalign*}
\end{xmp}

\begin{teo}[Законы Кеплера]~
\begin{enumerate}
\item Планеты движутся по эллипсам, в одном из фокусов которых находится солнце.
\item Радиус-вектор планеты заметает равные площади за равные промужутки времени.
\item $\frac{T^2}{a^3} = const$ (где $a$ ---большая полуось эллипса) для планет из одной системы.
\end{enumerate}
\end{teo}
\begin{proof}
TODO
\end{proof}
  \input{2017-11-08}
\end{document}
