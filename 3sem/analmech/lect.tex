\documentclass{article}

%\usepackage[a4paper, total={6in, 8in}]{geometry}

\usepackage{amsmath}
\usepackage{amsfonts}
\usepackage{amssymb}
\usepackage[T1, T2A]{fontenc}
\usepackage[utf8]{inputenc}
\usepackage[english, russian]{babel}
\usepackage{graphics}
\usepackage{amsthm}

\newtheorem*{df}{Определение}
\newtheorem{teo}{Теорема}
\newtheorem{lem}{Лемма}
\newtheorem{prp}{Предложение}
\newtheorem{hyp}{Предположение}
\newtheorem{ass}{Утверждение}
\newtheorem*{cor}{Следствие}
\newtheorem*{ntc}{Замечание}
\newtheorem*{xmp}{Пример}

\renewcommand\qedsymbol{$\blacksquare$}

\newcommand{\lb}{\left(}
\newcommand{\rb}{\right)}


\setcounter{secnumdepth}{0}

\author{Муницина Валерия Александровна}
\title{Аналитическая механика}

\begin{document}
  \begin{titlepage}
  \maketitle
  \begin{center}
  {\itshape\footnotesize Набор: Александр Валентинов}

  {\itshape\footnotesize Об ошибках писать: vk.com/valentiay}
  \end{center}
  \tableofcontents
  \vfill
  \end{titlepage}

  \section{Кинематика точки}
  \begin{df}
  Материальная точка - точка, размером которой можно пренебречь.
  \end{df}
  
  \noindent Мы будем полагать, что время меняется равномерно и непрерывно.
  \begin{center}  
  \begin{picture}(100, 100)
  \put(40,40){\vector(1,0){50}} %
  \put(40,40){\vector(0,1){50}} %
  \put(40,40){\vector(-1, -1){25}} %
  \put(40,40){\vector(2,1){40}} %
  \put(92,42){$x$} %
  \put(42,92){$z$} %
  \put(8,8){$y$} %
  \put(82,62){$\vec{r}$} %
  \qbezier(50,70)(60,80)(80,60) %
  \qbezier(80,60)(85,50)(90,50) %
  \end{picture}
  \end{center}
  \subsection{Векторное описание движения}
  Зависимость координат от времени назовем законом движения.
  $$ \vec{r} = \vec{r}(t) \in C^2 $$
  \begin{df}
  $ \gamma = \{ \vec{r}(t),~ t \in (0,~ +\infty) \} $ - траектория
  \end{df}
  $$ \vec{v} = \frac{d\vec{r}}{dt} $$
  $$ \vec{w} = \frac{d\vec{v}}{dt} = \frac{d^2\vec{r}}{dt^2} $$
  \subsection{Декартовы координаты}
  $$ \vec{r}(t) = x(t)\vec{e_x} + y(t)\vec{e_y} + z(t)\vec{e_z} $$
  $$ \vec{v}(t) = \dot x(t)\vec{e_x} + \dot y(t)\vec{e_y} + \dot z(t)\vec{e_z} $$
  $$ \vec{w}(t) = \ddot x(t)\vec{e_x} + \ddot y(t)\vec{e_y} + \ddot z(t)\vec{e_z} $$
  \subsection{Движение по окружности}
  $$ 
  \begin{cases}
   x = R \cos \varphi \\
   y = R \sin \varphi
  \end{cases} 
  $$

  $$ 
  \begin{cases}
  \dot x = -R \sin \varphi \cdot \dot \varphi \\
  \dot y = R \cos \varphi \cdot \dot \varphi 
  \end{cases}
  $$
  
  $$ 
  \begin{cases}
  \ddot x = -R \cos \varphi \cdot  \dot \varphi^2 - R \sin \varphi \cdot \ddot \varphi \\
  \ddot y = -R \sin \varphi \cdot  \dot \varphi^2 + R \cos \varphi \cdot \ddot \varphi 
  \end{cases}
  $$
  
  \begin{center}
  \begin{picture}(100,100)
  \put(50,50){\circle{50}} %
  \put(50,50){\line(1,1){14}} %
  \put(50,50){\line(-1,-1){14}} %
  \put(5,50){\vector(1,0){90}} %
  \put(97,52){$x$} %
  \put(50,5){\vector(0,1){90}} %
  \put(52,97){$y$} %
  \put(65,65){$\varphi(t)$} %
  \put(36,36){\vector(1,-1){10}} %
  \put(40,15){$\vec{\tau}$} %
  \put(36,36){\vector(1,1){10}} %
  \put(31,40){$\vec{n}$} %
  \end{picture}
  \end{center}
  
  $$ \vec{v} = R\dot\varphi(-\sin \varphi \cdot \vec{e_x} + \cos \varphi \cdot\vec{e_y}) = R \dot\varphi \vec{r} $$
  $$ \vec{w} = R \ddot\varphi ( - \sin \varphi \cdot \vec{e_x} + \cos \varphi \cdot \vec{e_y}) + R \dot\varphi^2(-\cos \varphi \cdot \vec{e_x} - \sin \varphi \cdot {\vec{e_y}}) = R \ddot \varphi \vec{\tau} + R \dot \varphi^2 \vec{n}  $$
  
  $$ \vec{v} = R \dot\varphi \vec{\tau} = v \vec{\tau}$$
  $$ \vec{w} = R \ddot\varphi \vec{\tau} + R \dot \varphi^2 \vec{n} = \dot v \vec{\tau} + \frac{v^2}{R} \vec{n}$$
  
  \subsection{Естественное описание движения}
  Кривая задана параметрически естественным параметром $s$. $ ds = |\vec{dr}| \neq 0 $
  \begin{df}
  \begin{equation} 
  \label{tang}
  \vec{\tau} = \frac{d\vec{r}}{ds} = \dot{\vec r} \text{ - касательный вектор}
  \end{equation}
  \begin{equation}
  \label{normal}
  \vec{n} = \frac{\dot{\vec r}}{|\dot{\vec {\tau}}|} \text{ - вектор главной нормали }
  \end{equation}
  \begin{equation}    
  \vec{b} = [\vec{t}; \vec{n}] \text{ - вектор бинормали }
  \end{equation}  
  \end{df}
  
  \begin{ass} 
  $ \{\vec{\tau}, \vec{n}, \vec{b}\} $ - тройка ортогональных единичных векторов.
  \end{ass}
  \begin{proof}  
  \begin{gather}
  |\vec{\tau}| = \frac{|d\vec{r}|}{|ds|} = 1 \\
  |\vec{n}| = \frac{|\dot{\vec r}|}{|\dot{\vec{\tau}}|} = 1 \\
  |\vec{\tau}| = 1 \Rightarrow (\tau, \tau) = 1 \\
  (\dot{\vec {\tau}}, \vec{\tau}) + (\vec{\tau}, \dot{\vec {\tau}}) = 0 \\
  2 (\dot{\vec{\tau}}, \vec{\tau}) = 0 \Rightarrow \dot{\vec{\tau}} \perp \vec{\tau} \Rightarrow \vec{n} \perp \vec{\tau}
  \end{gather}
  
  \end{proof}
  Этот трехгранник называют репер Ферне. (Дарбу, сопровождающий трехгранник).
  
  \begin{teo} 
  $ \vec{v} = v \vec{\tau} $, $ \vec{w} = \dot v \vec{\tau} + \frac{v^2}{\rho} \vec{n} $, где $ v = \dot s $.
  \end{teo}
  \begin{proof}
  \begin{gather}
  \vec{v} = \frac{d\vec{r}}{dt} = \frac{d\vec{r}}{ds} \frac{ds}{dt} = v\vec{\tau} \\
  \dot{\vec {\tau}} = \frac{d\vec{\tau}}{ds} \frac{ds}{dt} = \vec{n}kv \text{, по формуле (\ref{normal})} \\
  \vec{w} = \dot{\vec {v}} = \dot v \vec{\tau} + v \dot{\vec {\tau}} = \dot v \vec{\tau} + v^2 k \vec{n} = \dot v \vec{\tau} + \frac{v^2}{\rho} \vec{n} 
  \end{gather}
  $$ \dot v \vec{\tau} \text{ - касательное ускорение} $$
  $$ \frac{v^2}{\rho} \vec{n} \text{ - нормальное ускорение } $$
  $$ \rho = \frac{1}{|\dot r|} \text{ - радиус кривизны} $$
  $$ k = | \vec{\ddot r} | \text{ - кривизна} $$
  $$ \vec{\ddot r} \text{ - вектор кривизны} $$
  
  \end{proof}
  
  \paragraph{Формулы Френеля:}
  $$ 
  \begin{cases}
  \vec{\tau}' = k \vec{n} \\
  \vec{n}' = - k\vec{\tau} + \varkappa \vec{b} \\
  \vec{b}' = -\varkappa\vec{n}
  \end{cases}
  $$
  где $\varkappa$ - коэффициент кручения.
  
  \begin{proof}
  $$ | \vec{n} | = 1 \Rightarrow (\vec{n}, \vec{n}) = 0 $$
  $$ \vec{n} \perp \vec{\tau} \Rightarrow (\vec{n}', \vec{\tau}) + (\vec{n}, \vec{\tau}') = 0 \Rightarrow (\vec{n}', \vec{\tau}) + k = 0 $$
  
  $$ \vec{b}' = [\vec{ \tau}', \vec{n}] + [\vec{\tau}, \vec{n}'] = [k\vec{n}, \vec{n}] + [\vec{\tau}, -k\vec{\tau} + \varkappa \vec{b}] = 0 + \varkappa[\vec{r}, \vec{b}] = -\varkappa\vec{n} $$
  \end{proof}
  \subsection{Ортогональные векторные координаты}
  
  \begin{gather}
  \vec{r} = \vec{r}(q_1(t), q_2(t), q_3(t)) \\
  \vec{v} = \dot{\vec {\tau}}= \sum \limits_{i = 1}^3 \frac{\partial\vec{r}}{\partial q_i} \dot q_i\\
  \vec{H_i} = \frac{\partial\vec{r}}{\partial q_i} = H_i \vec{e_i} \text{, где $H_i$ - коэффициенты Ламе.} \\ 
  \end{gather}
  \subsubsection{Геометрический смысл}
  $$ ds_i = H_i dq_i $$
  $s_i$ - длина дуги $i$-й к-ой линии.
  $$ H_i = \frac{\partial \vec{r}}{\partial q_i}  = \sqrt{\left(\frac{\partial x}{\partial q_i}\right) ^2 + \left(\frac{\partial y}{\partial q_i}\right) ^2 + \left(\frac{\partial z}{\partial q_i}\right) ^2} $$
  $$ \vec{v} = \sum\limits_{i=1}^3 H_i \dot{q_i} \vec{e_i},~~ v^2 = (\vec{v}, \vec{v}) = \sum H_i^2\dot{q_i^2} $$ 
  \begin{teo}
  Копоненты вектора ускорения в ортогональном криволинейном базисе определяются равенством:
  $$ w_i = \frac{1}{H_i}\left(\frac{d}{dt} \frac{\partial}{\partial \dot q_i} \left(\frac{v^2}{2}\right) - \frac{\partial}{\partial q_i} \left(\frac{v^2}{2} \right) \right) $$
  \end{teo}
  \begin{proof}
  \begin{gather}
(\vec{w}, \vec{H_i}) = \left(\frac{d\vec{v}}{dt}, \frac{\partial \vec{r}}{\partial q_i} \right) = \frac{d}{dt} \left(\vec{v}, \frac{\vec{r}}{\partial q_i}\right) - \left(\vec{v}, \frac{d}{dt} \frac{\partial \vec{r}}{\partial q_i} \right) \triangleq \\
1) ~ \frac{\partial \vec{r}}{\partial q_i} = \frac{\partial \vec{v}}{\partial q_i'} \text{ - из определения скорости} \\
2) ~ \frac{d}{dt} \left(\frac{\partial \vec{r}}{\partial q_i} \right) = \sum \limits_{j = 1}^3 \frac{\partial^2 \vec{r}}{\partial q_j \partial q_i} \dot q_j = \sum \limits_{j = 1}^3 \frac{\partial^2 \vec{r}}{\partial q_i \partial q_j} \dot q_j = \\ 
= \frac{\partial}{\partial q_i} \left( \frac{d\vec{r}}{dt} \right) = \frac{\partial \dot{\vec r}}{\partial q_i} = \frac{\partial \vec{v}}{\partial q_i} \\
\triangleq \frac{d}{dt} \left(\vec{v}, \frac{\partial \vec{v}}{\partial q_i} \right) - \left( \vec{v}, \frac{\partial \vec{v}}{\partial q_i} \right) = \frac{d}{dt} \frac{1}{2} \frac{\partial}{\partial q_i} (\vec{v}, \vec{v}) - \frac{1}{2} \frac{\partial}{\partial q_i} (\vec{v}, \vec{v}) = \\ 
= \frac{d}{dt} \frac{\partial}{\partial \dot q_i} \left(\frac{v^2}{2} \right) - \frac{\partial}{\partial q_i} \left(\frac{v^2}{2}\right) \\
w_i = (\vec{w}, \vec{e_i}) = \frac{1}{H_i}(\vec{w}, \vec{H_i})
  \end{gather}
  \end{proof}
  
  %2017-09-13
  
  \section{Кинематика твердого тела}
  \begin{df}
  Абсолютно твердым телом называется множество точек, расстояние между которыми не меняется со временем.
  
  $  \{ \vec{r_i}, i = \overline{1 \ldots n} ~~:~~|\vec{r_i} - \vec{r_j} | = C_{ij} = const ,~~ n \geqslant 3 \}$ 
  
  \end{df}
  
  $OXYZ$ - неподвижная система отсчета.
  
  $S\xi\eta\zeta$ - связаны с телом (движется).
 
  $$
  X = 
  \left(
  \begin{matrix} 
  (\vec{e_{\xi}}, \vec{e_{x}}) & 
  (\vec{e_{\xi}}, \vec{e_{y}}) & 
  (\vec{e_{\xi}}, \vec{e_{z}}) \\ 
  (\vec{e_{\eta}}, \vec{e_{x}}) & 
  (\vec{e_{\eta}}, \vec{e_{y}}) & 
  (\vec{e_{\eta}}, \vec{e_{z}}) \\  
  (\vec{e_{\zeta}}, \vec{e_{x}}) & 
  (\vec{e_{\zeta}}, \vec{e_{y}}) & 
  (\vec{e_{\xi}}, \vec{e_{\zeta}}) \\
  \end{matrix}
  \right)
  \text{ - матрица направляющих косинусов.}
  $$
 
  $$ \vec{AB} = x\vec{e_x} + y\vec{e_y} + z\vec{e_z} $$
  $$ \vec{AB} = \xi\vec{e_{\xi}} + \eta\vec{e_{\eta}} + \zeta\vec{e_{\zeta}} $$

  $$ X
  \left(
  \begin{matrix}
    x \\ y \\ z \\
  \end{matrix}
  \right)
  =
  \left(
  \begin{matrix}
  (\vec{e_{\xi}}, x\vec{e_x} + y \vec{e_y} + z \vec{e_z}) \\
  (\vec{e_{\eta}}, x\vec{e_x} + y \vec{e_y} + z \vec{e_z}) \\
  (\vec{e_{\zeta}}, x\vec{e_x} + y \vec{e_y} + z \vec{e_z}) \\
  \end{matrix}
  \right)
  = 
  \left(
  \begin{matrix}
  (\vec{e_{\xi}}, \vec{AB}) \\
  (\vec{e_{\eta}}, \vec{AB}) \\
  (\vec{e_{\zeta}}, \vec{AB}) \\
  \end{matrix}
  \right)
  =
  \left(
  \begin{matrix}
  \xi \\
  \eta \\
  \zeta \\
  \end{matrix}
  \right)
  =
  \vec{\rho}
  $$

  $$ \vec{\rho} = X \vec{r} $$
  
  \begin{ass}
  $X$ - ортогональная матрица.
  \end{ass}
  \begin{proof}
  $$ XX^T = X^TX = 
  \left(
  \begin{matrix}
  (\vec{e_{\xi}}, \vec{\xi}) & 
  (\vec{e_{\xi}}, \vec{\eta}) & 
  (\vec{e_{\xi}}, \vec{\zeta}) \\
  \vdots & \ddots & \vdots \\
   & \ldots &
  \end{matrix} 
  \right)
  = 0 $$
  Т.к. базис ортогональный.
  \end{proof} 
  
  $$
  \left(
  \begin{matrix}
  \vec{e_{\xi}} \\
  \vec{e_{\eta}} \\
  \vec{e_{\zeta}} \\
  \end{matrix}
  \right)  
  =  
  X
  \left(
  \begin{matrix}
  \vec{e_{x}} \\
  \vec{e_{y}} \\
  \vec{e_{z}} \\
  \end{matrix}
  \right)
  $$

  $$
  \left(
  \begin{matrix}
  \dot{\vec{e_{\xi}}} \\
  \dot{\vec{e_{\eta}}} \\
  \dot{\vec{e_{\zeta}}} \\  
  \end{matrix}
  \right)
  = 
  \dot{X}
  \left(
  \begin{matrix}
  \vec{e_{x}} \\
  \vec{e_{y}} \\
  \vec{e_{z}} \\
  \end{matrix}
  \right) =
  \underbrace{
  \dot{X} X^T
  }_{\Omega}
  \left(
  \begin{matrix}
  \vec{e_{\xi}} \\
  \vec{e_{\eta}} \\
  \vec{e_{\zeta}} \\
  \end{matrix}
  \right)
  =
  \Omega
  \left(
  \begin{matrix}
  \vec{e_{\xi}} \\
  \vec{e_{\eta}} \\
  \vec{e_{\zeta}} \\
  \end{matrix}
  \right) 
  $$

  $$ \Omega = \dot X X^T $$

  
  \begin{ass}
  $\Omega$ - кососимметрична.
  \end{ass}
  \begin{proof}
  $$ \Omega \Omega^2 = \dot X X^T + (\dot X X^T)T = \dot X X^T + X \dot {X^T} = \frac{d}{dt}(XX^T) = 
  \frac{d}{dt}(E) = 0 $$
  \end{proof}
  
  \begin{cor}
  $$ \Omega =
  \left(
  \begin{matrix}
  0 & \omega_{\zeta} & -\omega_{\eta} \\
  -\omega_{\zeta} & 0 & \omega_{\xi} \\
  \omega_{\eta} & -\omega_{\xi} & 0 \\
  \end{matrix}
  \right)
  \text{- Факт, который может быть законспектирован неправильно}
  $$
  \end{cor}
  
  \begin{df}
  $ \vec{\omega} = \omega_{\xi}\vec{e_{\xi}} + \omega_{\eta}\vec{e_{\eta}} + \omega_{\zeta}\vec{e_{\zeta}} $ - угловая скорость подвижного репера.
  \end{df}
  
  \subsection{Формулы Пуассона}
  \begin{ass}
  $$ \dot{\vec{e_i}} = [\vec{\omega}, \vec{e_i}],~~ i = \overline{1 \ldots 3} $$
  \end{ass}
  \begin{proof}
  $$
  \dot{\vec{e_{\xi}}} = \omega_{\zeta} \vec{e_{\eta}} - \omega_{\eta} \vec{e_{\zeta}} =
  \begin{vmatrix}
  \vec{e_{\xi}} & \vec{e_{\eta}} & \vec{e_{\zeta}} \\
  \omega_{\xi} & \omega_{\eta} & \omega_{\zeta} \\
  1 & 0 & 0 \\ 
  \end{vmatrix}
  =
  [\vec{\omega}, \vec{e_{\xi}}] 
  $$
  \end{proof}
  
  \begin{ass}
  $ \vec{\omega} = \vec{e_{\xi}}(\dot{\vec{e_{\eta}}}, \vec{e_{\zeta}}) + \vec{e_{\eta}}(\dot{\vec{e_{\zeta}}}, \vec{e_{\xi}}) + \vec{e_{\zeta}}(\dot{\vec{e_{\xi}}}, \vec{e_{\eta}}) $
  \end{ass}
  \begin{proof}
  $$ (\dot{\vec{e_{\xi}}}, \vec{e_{\eta}}) = \omega_{\zeta} $$
  $$ (\dot{\vec{e_{\eta}}}, \vec{e_{\zeta}}) = \omega_{\xi} $$
  $$ (\dot{\vec{e_{\zeta}}}, \vec{e_{\xi}}) = \omega_{\eta} $$
  \end{proof}
  
  \begin{ass}
  $ \vec{\omega} = \frac{1}{2} ([\vec{e_{\xi}}, \dot{\vec{e_{\xi}}}] + [\vec{e_{\eta}}, \dot{\vec{e_{\eta}}}] + [\vec{e_{\zeta}}, \dot{\vec{e_{\zeta}}}]) $
  \end{ass}
  \begin{proof}
  $$ \vec{\omega} 
  = \frac{1}{2} ([\vec{e_{\xi}}, \dot{\vec{e_{\xi}}}] + [\vec{e_{\eta}}, \dot{\vec{e_{\eta}}}] + [\vec{e_{\zeta}}, \dot{\vec{e_{\zeta}}}]) 
  = \frac{1}{2} ([\vec{e_{\xi}}, [\vec{\omega}, \vec{e_{\xi}}]] + [\vec{e_{\eta}}, [\vec{\omega}, \vec{e_{\eta}}]] + [\vec{e_{\zeta}}, [\vec{\omega}, \vec{e_{\zeta}}]]) = $$
  $$ = \frac{1}{2} \left( \vec{\omega}(\vec{e_{\xi}}, \vec{e_{\xi}}) - \vec{e_{\xi}}(\vec{\omega}, \vec{e_{\xi}}) + \vec{\omega}(\vec{e_{\eta}}, \vec{e_{\eta}}) - \vec{e_{\eta}}(\vec{\omega}, \vec{e_{\eta}}) + \vec{\omega}(\vec{e_{\zeta}}, \vec{e_{\zeta}}) - \vec{e_{\zeta}}(\vec{\omega}, \vec{e_{\zeta}}) \right) = $$ 
  $$ = \frac{1}{2}(3\vec{\omega} - \vec{\omega}) = \vec{\omega} $$
  \end{proof}
  
  \begin{xmp}
  Угловая скорость репера Френеля.
  $$ 
  \begin{cases}
  \vec{\tau}' = k \vec{n} \\
  \vec{n}' = - k\vec{\tau} + \varkappa \vec{b} \\
  \vec{b}' = -\varkappa\vec{n}
  \end{cases}
  $$
  
  $$
  \begin{cases}
  \dot{\vec{\tau}} = \frac{d\vec{\tau}}{ds} \dot s \\
  \dot{\vec{n}} = \frac{d\vec{n}}{ds} \dot s \\
  \dot{\vec{b}} = \frac{d\vec{b}}{ds} \dot s \\
  \end{cases} 
  $$
  
  $$ \vec{\omega} = \vec{\tau}(\dot s (-k\vec{\tau} + \varkappa\vec{b}), \vec{b}) + \vec{n}(\dot s (-\varkappa\vec{n}, \vec{\tau}) + \vec{b}(\dot s(k \vec{n}), \vec{n}) = \dot s (\varkappa\vec{\tau} + k\vec{b}) $$
  \end{xmp}
  
  \begin{df}
  Угловой скоростью твердого тела называется угловая скорость подвижного репера, с ним свзязанного.
  \end{df}
 
  \subsection{Формула распределения скоростей точек твердого тела}
  $ \vec{v_B} = \vec{v_A} + [\vec{\omega}, \vec{AB}] $
  \begin{proof}
  $$ \vec{AB} = \xi \vec{e_{\xi}} + \eta \vec{e_{\eta}} + \zeta \vec{e_{\zeta}} $$
  $$ \dot{\vec{AB}} = \xi \dot{\vec{e_{\xi}}} + \eta \dot{\vec{e_{\eta}}} + \zeta \dot{\vec{e_{\zeta}}},~~ \dot{\xi} = \dot{\eta} = \dot{\zeta} = 0 $$
  $$ \dot{\left(\vec{r_B} - \vec{r_A}\right)}~~ = \xi[\vec{\omega}, \vec{e_{\xi}}] + \eta[\vec{\omega}, \vec{e_{\eta}}] + \zeta[\vec{\omega}, \vec{e_{\zeta}}] $$ 
  $$ \dot{\vec{r_1}} - \dot{\vec{r_2}} = [\vec{\omega}, \xi \vec{e_{\xi}} + \eta \vec{e_{\eta}} + \zeta \vec{e_{\zeta}}] $$
  $$ \vec{v_B} = \vec{v_A} + [\vec{\omega}, \vec{AB}] $$  
  \end{proof}
  
  \begin{cor}
  $S\xi\eta\zeta \rightarrow \vec{\omega}$, $S'\xi'\eta'\zeta' \rightarrow \vec{\omega}'$
  $$ 
  \left.
  \begin{array}{ccc}
  \vec{v_B} = \vec{v_A} + [\vec{\omega}, \vec{AB}] \\
  \vec{v_B} = \vec{v_A} + [\vec{\omega'}, \vec{AB}]
  \end{array}
  \right|
  [\vec{\omega} - \vec{\omega}', \vec{AB}] = 0;~ \forall A, B \text{ в абсолютно твердом теле} \Rightarrow
  $$
  $$ \Rightarrow \vec{\omega} - \vec{\omega}' = 0 \Rightarrow \boxed{\vec{\omega} = \vec{\omega}'} $$
  \end{cor}
  
  \begin{ass}(Формула Ривальса) $ \vec{w_B} = \vec{w_A} + [\vec{\varepsilon}, \vec{AB}] + [\vec{\omega}, [\omega, \vec{AB}]] $.
  \end{ass}
  \begin{proof}
  $$ \vec{v_B} = \vec{v_A} + [\vec{\omega}, \vec{AB}] $$
  $$ \dot{\vec{v_B}} = \dot{\vec{v_A}} + [\dot{\vec{\omega}}, \vec{AB}] + [\vec{\omega}, \dot{\vec{r_B} - \vec{r_A}}~]  $$
  $$ \vec{w_B} = \vec{w_A} + [\vec{\varepsilon}, \vec{AB}] + [\vec{\omega}, [\vec{\omega}, \vec{AB}]]  $$
  $$ [\vec{\varepsilon}, \vec{AB}] \text{ - вращательное ускорение,~~} [\vec{\omega}, [\vec{\omega}, \vec{AB}]] \text{ - осестремительное ускорение} $$ 
  \end{proof}
  
  \subsubsection{Геометрический смысл}
  $ \vec{w} = [\vec{\omega}, [\vec{\omega}, \vec{AB}]] = \vec{\omega} (\vec{\omega}, \vec{AB}) - \vec{AB} \omega^2 = \omega^2 ( \vec{e_{\omega}}(\vec{AB}, \vec{e_{\omega}}) - \vec{AB}) $
  
  $ | \vec{w_{\textbf{ос}}} |= \omega^2 \rho(B, l) $
  
  \begin{ass}
  Проекции скоростей двух точек твердого тела на прямую, их соединяющую, равны.
  \end{ass}
  \begin{proof}
  $$ \vec{v_B} = \vec{v_A} + [\vec{\omega}, \vec{AB}] $$
  $$ (\vec{v_B}, \vec{AB}) = (\vec{v_A}, \vec{AB}) + ([\vec{\omega}, \vec{AB}], \vec{AB}) $$
  $$ v_B \cos \beta = v_A \cos \alpha $$
  \end{proof}
  \begin{ntc}
  Аналогичная теорема для ускорений не верна.
  \end{ntc}
  
  \section{Классификация движения твердого тела}
  
  \subsection{Поступательное}
  \begin{df}
  Такое движение твердого тела, при котором угловая скорость равна нулю.
  \end{df}
  $$ \vec{v_B} \equiv \vec{v_A} $$
  $$ \vec{w_B} \equiv \vec{v_A} $$
  \subsubsection{Мгновенное поступательное движение}
  $ \exists t : \vec{\omega}(t) = 0,~~ \vec{\varepsilon}(t) \neq 0 $
  
  \subsection{Вращательное движение (вращение вокруг неподвижной оси)}
  %$ \exists A, B : \vec{v_A} = \vec{v_B} = 0 $
  
  %$ \vec{v_B} = \vec{v_A} + [\vec{\omega}, \vec{AB}], \vec{v_A} = \vec{v_B} = 0 \Rightarrow [\omega, \vec{AB}] = 0 \Rightarrow \omega \parallel \vec{AB} $
  %$\forall M \in l : \vec{v_M} = 0 \text{, $l$ - ось} $
  TODO
\end{document}