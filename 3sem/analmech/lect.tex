\documentclass{article}

%\usepackage[a4paper, total={6in, 8in}]{geometry}

\usepackage{amsmath}
\usepackage{amsfonts}
\usepackage{amssymb}
\usepackage[T1, T2A]{fontenc}
\usepackage[utf8]{inputenc}
\usepackage[english, russian]{babel}
\usepackage{graphics}
\usepackage{amsthm}

\newtheorem*{df}{Определение}
\newtheorem{teo}{Теорема}
\newtheorem{lem}{Лемма}
\newtheorem{prp}{Предложение}
\newtheorem{hyp}{Предположение}
\newtheorem{ass}{Утверждение}
\newtheorem*{cor}{Следствие}
\newtheorem*{ntc}{Замечание}

\renewcommand\qedsymbol{$\blacksquare$}

\newcommand{\lb}{\left(}
\newcommand{\rb}{\right)}

\author{Муницина Валерия Александровна}
\title{Аналитическая механика}

\begin{document}
  \maketitle
  \section*{Кинематика точки}
  \begin{df}
  Материальная точка - точка, размером которой можно пренебречь.
  \end{df}
  
  \noindent Мы будем полагать, что время меняется равномерно и непрерывно.
  \begin{center}  
  \begin{picture}(100, 100)
  \put(40,40){\vector(1,0){50}} %
  \put(40,40){\vector(0,1){50}} %
  \put(40,40){\vector(-1, -1){25}} %
  \put(40,40){\vector(2,1){40}} %
  \put(92,42){$x$} %
  \put(42,92){$z$} %
  \put(8,8){$y$} %
  \put(82,62){$\overrightarrow{r}$} %
  \qbezier(50,70)(60,80)(80,60) %
  \qbezier(80,60)(85,50)(90,50) %
  \end{picture}
  \end{center}
  \subsection*{Векторное описание движения}
  Зависимость координат от времени назовем законом движения.
  $$ \overrightarrow{r} = \overrightarrow{r}(t) \in C^2 $$
  \begin{df}
  $ \gamma = \{ \overrightarrow{r}(t),~ t \in (0,~ +\infty) \} $ - траектория
  \end{df}
  $$ \overrightarrow{v} = \frac{d\overrightarrow{r}}{dt} $$
  $$ \overrightarrow{w} = \frac{d\overrightarrow{v}}{dt} = \frac{d^2\overrightarrow{r}}{dt^2} $$
  \subsection*{Декартовы координаты}
  $$ \overrightarrow{r}(t) = x(t)\overrightarrow{e_x} + y(t)\overrightarrow{e_y} + z(t)\overrightarrow{e_z} $$
  $$ \overrightarrow{v}(t) = \dot x(t)\overrightarrow{e_x} + \dot y(t)\overrightarrow{e_y} + \dot z(t)\overrightarrow{e_z} $$
  $$ \overrightarrow{w}(t) = \ddot x(t)\overrightarrow{e_x} + \ddot y(t)\overrightarrow{e_y} + \ddot z(t)\overrightarrow{e_z} $$
  \subsection*{Движение по окружности}
  $$ 
  \begin{cases}
   x = R \cos \varphi \\
   y = R \sin \varphi
  \end{cases} 
  $$

  $$ 
  \begin{cases}
  \dot x = -R \sin \varphi \cdot \dot \varphi \\
  \dot y = R \cos \varphi \cdot \dot \varphi 
  \end{cases}
  $$
  
  $$ 
  \begin{cases}
  \ddot x = -R \cos \varphi \cdot  \dot \varphi^2 - R \sin \varphi \cdot \ddot \varphi \\
  \ddot y = -R \sin \varphi \cdot  \dot \varphi^2 + R \cos \varphi \cdot \ddot \varphi 
  \end{cases}
  $$
  
  \begin{center}
  \begin{picture}(100,100)
  \put(50,50){\circle{50}} %
  \put(50,50){\line(1,1){14}} %
  \put(50,50){\line(-1,-1){14}} %
  \put(5,50){\vector(1,0){90}} %
  \put(97,52){$x$} %
  \put(50,5){\vector(0,1){90}} %
  \put(52,97){$y$} %
  \put(65,65){$\varphi(t)$} %
  \put(36,36){\vector(1,-1){10}} %
  \put(40,15){$\overrightarrow{\tau}$} %
  \put(36,36){\vector(1,1){10}} %
  \put(31,40){$\overrightarrow{n}$} %
  \end{picture}
  \end{center}
  
  $$ \overrightarrow{v} = R\dot\varphi(-\sin \varphi \cdot \overrightarrow{e_x} + \cos \varphi \cdot\overrightarrow{e_y}) = R \dot\varphi \overrightarrow{r} $$
  $$ \overrightarrow{w} = R \ddot\varphi ( - \sin \varphi \cdot \overrightarrow{e_x} + \cos \varphi \cdot \overrightarrow{e_y}) + R \dot\varphi^2(-\cos \varphi \cdot \overrightarrow{e_x} - \sin \varphi \cdot {\overrightarrow{e_y}}) = R \ddot \varphi \overrightarrow{\tau} + R \dot \varphi^2 \overrightarrow{n}  $$
  
  $$ \overrightarrow{v} = R \dot\varphi \overrightarrow{\tau} = v \overrightarrow{\tau}$$
  $$ \overrightarrow{w} = R \ddot\varphi \overrightarrow{\tau} + R \dot \varphi^2 \overrightarrow{n} = \dot v \overrightarrow{\tau} + \frac{v^2}{R} \overrightarrow{n}$$
  
  \subsection*{Естественное описание движения}
  Кривая задана параметрически естественным параметром $s$. $ ds = |\overrightarrow{dr}| \neq 0 $
  \begin{df}
  \begin{equation} 
  \label{tang}
  \overrightarrow{\tau} = \frac{d\overrightarrow{r}}{ds} = \overrightarrow{\dot r} \text{ - касательный вектор}
  \end{equation}
  \begin{equation}
  \label{normal}
  \overrightarrow{n} = \frac{\overrightarrow{\dot r}}{|\overrightarrow{\dot \tau}|} \text{ - вектор главной нормали }
  \end{equation}
  \begin{equation}    
  \overrightarrow{b} = [\overrightarrow{t}; \overrightarrow{n}] \text{ - вектор бинормали }
  \end{equation}  
  \end{df}
  
  \begin{ass} 
  $ \{\overrightarrow{\tau}, \overrightarrow{n}, \overrightarrow{b}\} $ - тройка ортогональных единичных векторов.
  \end{ass}
  \begin{proof}  
  \begin{gather}
  |\overrightarrow{\tau}| = \frac{|d\overrightarrow{r}|}{|ds|} = 1 \\
  |\overrightarrow{n}| = \frac{|\overrightarrow{\dot r}|}{|\overrightarrow{\dot \tau}|} = 1 \\
  |\overrightarrow{\tau}| = 1 \Rightarrow (\tau, \tau) = 1 \\
  (\overrightarrow{\dot \tau}, \overrightarrow{\tau}) + (\overrightarrow{\tau}, \overrightarrow{\dot \tau} = 0 \\
  2 (\overrightarrow{\dot \tau}, \overrightarrow{\tau}) = 0 \Rightarrow \overrightarrow{\dot \tau} \perp \overrightarrow{\tau} \Rightarrow \overrightarrow{n} \perp \overrightarrow{\tau}
  \end{gather}
  
  \end{proof}
  Этот трехгранник называют репер Ферне. (Дарбу, сопровождающий трехгранник).
  
  \begin{teo} 
  $ \overrightarrow{v} = v \overrightarrow{\tau} $, $ \overrightarrow{w} = \dot v \overrightarrow{\tau} + \frac{v^2}{\rho} \overrightarrow{n} $, где $ v = \dot s $.
  \end{teo}
  \begin{proof}
  \begin{gather}
  \overrightarrow{v} = \frac{d\overrightarrow{r}}{dt} = \frac{d\overrightarrow{r}}{ds} \frac{ds}{dt} = v\overrightarrow{\tau} \\
  \overrightarrow{\dot \tau} = \frac{d\overrightarrow{\tau}}{ds} \frac{ds}{dt} = \overrightarrow{n}kv \text{, по формуле (\ref{normal})} \\
  \overrightarrow{w} = \overrightarrow{\dot v} = \dot v \overrightarrow{\tau} + v \overrightarrow{\dot \tau} = \dot v \overrightarrow{\tau} + v^2 k \overrightarrow{n} = \dot v \overrightarrow{\tau} + \frac{v^2}{\rho} \overrightarrow{n} 
  \end{gather}
  $$ \dot v \overrightarrow{\tau} \text{ - касательное ускорение} $$
  $$ \frac{v^2}{\rho} \overrightarrow{n} \text{ - нормальное ускорение } $$
  $$ \rho = \frac{1}{|\dot r|} \text{ - радиус кривизны} $$
  $$ k = | \overrightarrow{\ddot r} | \text{ - кривизна} $$
  $$ \overrightarrow{\ddot r} \text{ - вектор кривизны} $$
  
  \end{proof}
  
  \paragraph{Формулы Френе:}
  $$ 
  \begin{cases}
  \overrightarrow{\tau}' = k \overrightarrow{n} \\
  \overrightarrow{n}' = - k\overrightarrow{\tau} + \varkappa \overrightarrow{b} \\
  \overrightarrow{b}' = -\varkappa\overrightarrow{n}
  \end{cases}
  $$
  где $\varkappa$ - коэффициент кручения.
  
  \begin{proof}
  $$ | \overrightarrow{n} | = 1 \Rightarrow (\overrightarrow{n}, \overrightarrow{n}) = 0 $$
  $$ \overrightarrow{n} \perp \overrightarrow{\tau} \Rightarrow (\overrightarrow{n}', \overrightarrow{\tau}) + (\overrightarrow{n}, \overrightarrow{\tau}') = 0 \Rightarrow (\overrightarrow{n}', \overrightarrow{\tau}) + k = 0 $$
  
  $$ \overrightarrow{b}' = [\overrightarrow{ \tau}', \overrightarrow{n}] + [\overrightarrow{\tau}, \overrightarrow{n}'] = [k\overrightarrow{n}, \overrightarrow{n}] + [\overrightarrow{\tau}, -k\overrightarrow{\tau} + \varkappa \overrightarrow{b}] = 0 + \varkappa[\overrightarrow{r}, \overrightarrow{b}] = -\varkappa\overrightarrow{n} $$
  \end{proof}
  \subsection*{Ортогональные векторные координаты}
  
  \begin{gather}
  \overrightarrow{r} = \overrightarrow{r}(q_1(t), q_2(t), q_3(t)) \\
  \overrightarrow{v} = \overrightarrow{\dot \tau}= \sum \limits_{i = 1}^3 \frac{\partial\overrightarrow{r}}{\partial q_i} \dot q_i\\
  \overrightarrow{H_i} = \frac{\partial\overrightarrow{r}}{\partial q_i} = H_i \overrightarrow{e_i} \text{, где $H_i$ - коэффициенты Ламе.} \\ 
  \end{gather}
  \subsubsection*{Геометрический смысл}
  $$ ds_i = H_i dq_i $$
  $s_i$ - длина дуги $i$-й к-ой линии.
  $$ H_i = \frac{\partial \overrightarrow{r}}{\partial q_i}  = \sqrt{\left(\frac{\partial x}{\partial q_i}\right) ^2 + \left(\frac{\partial y}{\partial q_i}\right) ^2 + \left(\frac{\partial z}{\partial q_i}\right) ^2} $$
  $$ \overrightarrow{v} = \sum\limits_{i=1}^3 H_i \dot{q_i} \overrightarrow{e_i},~~ v^2 = (\overrightarrow{v}, \overrightarrow{v}) = \sum H_i^2\dot{q_i^2} $$ 
  \begin{teo}
  Копоненты вектора ускорения в ортогональном криволинейном базисе определяются равенством:
  $$ w_i = \frac{1}{H_i}\left(\frac{d}{dt} \frac{\partial}{\partial \dot q_i} \left(\frac{v^2}{2}\right) - \frac{\partial}{\partial q_i} \left(\frac{v^2}{2} \right) \right) $$
  \end{teo}
  \begin{proof}
  \begin{gather}
(\overrightarrow{w}, \overrightarrow{H_i}) = \left(\frac{d\overrightarrow{v}}{dt}, \frac{\partial \overrightarrow{r}}{\partial q_i} \right) = \frac{d}{dt} \left(\overrightarrow{v}, \frac{\overrightarrow{r}}{\partial q_i}\right) - \left(\overrightarrow{v}, \frac{d}{dt} \frac{\partial \overrightarrow{r}}{\partial q_i} \right) \triangleq \\
1) ~ \frac{\partial \overrightarrow{r}}{\partial q_i} = \frac{\partial \overrightarrow{v}}{\partial q_i'} \text{ - из определения скорости} \\
2) ~ \frac{d}{dt} \left(\frac{\partial \overrightarrow{r}}{\partial q_i} \right) = \sum \limits_{j = 1}^3 \frac{\partial^2 \overrightarrow{r}}{\partial q_j \partial q_i} \dot q_j = \sum \limits_{j = 1}^3 \frac{\partial^2 \overrightarrow{r}}{\partial q_i \partial q_j} \dot q_j = \\ 
= \frac{\partial}{\partial q_i} \left( \frac{d\overrightarrow{r}}{dt} \right) = \frac{\partial \overrightarrow{\dot r}}{\partial q_i} = \frac{\partial \overrightarrow{v}}{\partial q_i} \\
\triangleq \frac{d}{dt} \left(\overrightarrow{v}, \frac{\partial \overrightarrow{v}}{\partial q_i} \right) - \left( \overrightarrow{v}, \frac{\partial \overrightarrow{v}}{\partial q_i} \right) = \frac{d}{dt} \frac{1}{2} \frac{\partial}{\partial q_i} (\overrightarrow{v}, \overrightarrow{v}) - \frac{1}{2} \frac{\partial}{\partial q_i} (\overrightarrow{v}, \overrightarrow{v}) = \\ 
= \frac{d}{dt} \frac{\partial}{\partial \dot q_i} \left(\frac{v^2}{2} \right) - \frac{\partial}{\partial q_i} \left(\frac{v^2}{2}\right) \\
w_i = (\overrightarrow{w}, \overrightarrow{e_i}) = \frac{1}{H_i}(\overrightarrow{w}, \overrightarrow{H_i})
  \end{gather}
  \end{proof}
\end{document}

