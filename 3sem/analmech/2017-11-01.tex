\section{Движение в центральном поле}
\subsection{Законы сохранения}
В центральном поле
\[
	m \ddot{ \v r} = \v F,\quad \v F = F(r) \frac{\v r}{r}
\]

\paragraph*{Закон сохранения энергии:}
\[
	\Pi = -\int F(r) dr,~~~ \frac{\partial \Pi}{\partial t} = 0 \Rightarrow T + \Pi = h = const
\]
\paragraph*{Закон сохранения кинетического момента:}
\[
	\v M_O = \left[ \v r, F(r) \frac{\v r}{r} \right] = 0 \Rightarrow \dot{\v k}_O = 0 \Rightarrow \v k_O = [\v r, \v m\v v] = \v k = const
\]
\begin{cor}
Траектория точки в центральном поле всегда является плоской кривой.
\end{cor}
\begin{proof}
\[
	[\v r, m\v v] = \v k \perp \alpha \Rightarrow \v r \in \alpha \quad \forall t,\,\alpha = const
\]
\end{proof}

\begin{cor}
\[
	r^2 \dot \varphi = c = const
\]
\end{cor}
\begin{proof}
\[
	| \v k | = |[ \v r, m \v v]| = |[r \v e_r,\, m (\dot {r } \v e_r + r \dot \varphi \v e_\varphi)] | = mr^2|\dot \varphi||\v e_z| = const \Rightarrow r^2\dot \varphi = const
\]
\end{proof}

\paragraph*{Геометрический смысл}
\begin{flalign*}
& S = \iint dS = \int\limits_{\varphi_0}^\varphi d\varphi\int\limits_0^{r(\varphi)} rdr = \int\limits_{\varphi_0}^\varphi \frac{r^2(\varphi)}{2}d \varphi &\\
& \dot S = \frac{dS}{d\varphi}, \quad \dot \varphi = \frac{r^2}{2}\varphi = \frac{c}{2} = const &\\
& \sigma = \dot s = \frac{c}{2} \text{ --- секториальная скорость}
\end{flalign*}

\subsection{Формулы Бине}

\begin{teo}[Формулы Бине]
При движении точи в центральном поле справедливы следующие равенства:
\[
	v^2 = c^2 \left( \left[ \frac{d}{d\varphi}\left( \frac{1}{r} \right) \right]^2 + \frac{1}{r^2} \right)
\]
\[
	F = - \frac{mc^2}{r^2} \left( \frac{d^2}{d\varphi^2}\left( \frac{1}{r} \right) + \frac{1}{r}\right)
\]
\end{teo}
\begin{proof}
\begin{flalign*}
& v^2 = \dot r^2 + r^2 \dot \varphi^2 &\\
& \v w = (\ddot r - r \dot \varphi^2)\v e^r + (r \ddot \varphi + 2\dot r \dot \varphi)\v e_\varphi &\\
& m \v w = F \v e_r \quad 
\begin{cases}
m(\ddot r - r \dot \varphi) = F \\
r \ddot \varphi + 2 \dot r \dot \varphi = 0 \Rightarrow \frac{d}{dt}(r^2 \dot \varphi) = 0 \\
\end{cases} &\\
& \dot r = \frac{dr}{d\varphi} \quad \dot \varphi = \frac{dr}{d\varphi}\frac{c}{r^2} = -c \frac{d}{d\varphi}\left( \frac{1}{r} \right) &\\
& \ddot r = \frac{d \dot r}{d \varphi}\dot \varphi = -\frac{c^2}{r^2} \frac{d^2}{d\varphi^2}\left( \frac{1}{r} \right) &\\
& v^2 = c^2 \left[ \frac{d}{d\varphi}\left( \frac{1}{r} \right) \right]^2 + r^2 \frac{c^2}{r^4} = c^2 \left( \left[ \frac{d}{d\varphi} \left( \frac{1}{r} \right)\right]^2 + \frac{1}{r^2} \right) &\\
& F = - \frac{mc^2}{r^2} \left( \frac{d^2}{d\varphi^2}\left( \frac{1}{r} \right) + \frac{1}{r}\right) &\\
\end{flalign*}
\end{proof}

\noindent Определим траекторию.
\begin{flalign*}
& T + \Pi = h, \quad T = \frac{m}{2}v^2 &\\
& \frac{mc^2}{2} \left[ \frac{d}{d\varphi}\left( \frac{1}{r} \right) \right]^2 + \underbrace{\frac{mc^2}{2r^2} + \Pi(r)}_{\Pi_c(r)} = h &\\
& \pm \sqrt{\frac{mc^2}{2}}\frac{d}{d\varphi}\left( \frac{1}{r} \right) = \sqrt{h - \Pi_c(r)} &\\
& \text{Замена: } \frac{1}{r} = u \quad \pm \sqrt{\frac{mc^2}{2}} \int\limits_{1/r_0}^{1/r} \frac{du}{\sqrt{h - \Pi_c(u)}} = \varphi - \varphi_0 \Rightarrow r(\varphi) &\\
& \dot \varphi = \frac{c}{r^2(\varphi)} \Rightarrow \int\limits_{\varphi_0}^\varphi r^2(\varphi) d \varphi = \int\limits_{t_0}^t cdt = c(t - t_0)
\end{flalign*}

\subsection{Движение точки в центральном гравитационном поле}
\[
	F = -\gamma \frac{mM}{r^2}, \quad \Pi(r) = -\gamma \frac{mM}{r}
\]
\begin{flalign*}
\varphi & = \pm \sqrt{\frac{mc^2}{2}}\int \frac{du}{\sqrt{h - m\frac{c^2}{2u^2} + \gamma mM u}} = \pm \int\frac{du}{\sqrt{\frac{2h}{mc^2} - u^2 + \frac{2\gamma M}{c^2}u}} = &\\
& = \pm \int \frac{du}{\sqrt{\frac{2h}{mc^2} + \frac{\gamma^2M^2}{c^4} - \left( u - \frac{\gamma M}{c^2} \right)^2}} = \pm \arccos \frac{\frac{1}{r} - \frac{\gamma M}{c^2}}{\sqrt{\frac{2h}{mc^2} + \frac{\gamma^2M^2}{c^4}}} + \varphi_0 &\\
& \frac{1}{r} = \frac{\gamma M}{c^2} + \sqrt{\frac{2h}{mc^2} + \frac{\gamma^2 M}{c^4}}\cos(\varphi - \varphi_0) &\\
& \frac{c^2}{\gamma m} = p, \quad \sqrt{\frac{2h}{mc^2}p^2 + 1} = e \Rightarrow r = \frac{p}{1 + e\cos(\varphi - \varphi_0)}
\end{flalign*}
То есть $\varphi_0$ зависит от $c$ и $h$.

\begin{ntc}
$\varphi_0 = 0 \quad (\varphi' = \varphi - \varphi_0)$
\end{ntc}

\begin{ass}
Траектория точки в центральном гравитационном поле является коническим сечением.
\begin{itemize}
\item $e = 0$: $\left( h^* := h = - \frac{mc^2}{2p^2} = - \frac{m\gamma^2M^2}{2c^2} \right)$ --- окружность.
\item $0 < e < 1$: $\left( h^* < h < 0 \right)$ --- эллипс.
\item $e = 1$: $\left( h = 0 \right)$ --- парабола.
\item $e > 1$: $\left( h > 0 \right)$ --- гипербола.
\end{itemize}
\end{ass}

\begin{xmp}[Первая космическая скорость]
\begin{flalign*}
& v_1 =\; ? &\\
& \frac{mv^2}{2} - \gamma \frac{mM}{R} = -\frac{m\gamma^2M^2}{2c^2} = - \frac{m\gamma^2 M^2}{2R^2v_1^2} &\\
& c = R^2 \dot \varphi = R v_1 \text{ (окружность)} &\\
& v_1^2 - \frac{2\gamma M}{R} + \frac{\gamma^2 M^2}{R^2v_1^2} = 0 &\\
& \left( v_1 - \frac{\gamma M}{R v_1} \right)^2 = 0 \Rightarrow v_1 = \sqrt{\frac{\gamma M}{R}} &\\
\end{flalign*}
\end{xmp}

\begin{xmp}[Вторая космическая скорость]
\begin{flalign*}
& \frac{mv_2^2}{2} - \frac{\gamma mM}{R} = 0 \Rightarrow v_2^2 = \frac{2\gamma M}{R} &\\
\end{flalign*}
\end{xmp}

\begin{teo}[Законы Кеплера]~
\begin{enumerate}
\item Планеты движутся по эллипсам, в одном из фокусов которых находится солнце.
\item Радиус-вектор планеты заметает равные площади за равные промужутки времени.
\item $\frac{T^2}{a^3} = const$ (где $a$ ---большая полуось эллипса) для планет из одной системы.
\end{enumerate}
\end{teo}
\begin{proof}
TODO
\end{proof}